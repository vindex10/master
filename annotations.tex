%!TEX root = ananyev-bachelor.tex
\begin{titlepage}
    \begin{otherlanguage}{english}
    \fontspec{Liberation Serif}[Ligatures=TeX]
    \section*{Annotation}
    \textbf{Ananyev V.O.} ``Application of Light by Light sumrules to charmonia and bottomonia radiative transitions'' \\
    {\itshape Master thesis on speciality 8.04020304 --- nuclear and high energy physics, specialization ``quantum field theory''. --- Taras Shevchenko National University of Kyiv, Faculty of Physics, Department of Quantum Field Theory. --- Kyiv, 2018.}

    \noindent {\bfseries Supervisor:} Prof. Dr. Marc Vanderhaeghen, Institute of Nuclear Physics
Johannes Gutenberg-Universität Mainz.

        Summation rules present a powerful approach to effects beyond non-relativistic quark model, connecting them to low-energy experimental data. In this work we defined main milestones and marked pitfalls on the way to establishing a sustainable channel between the frontiers. We excellently reproduced mass spectrum of charmonium and bottomonium spectrum and provided some ways of error controlling while looking for binding energies. We also developed a theoretical background for calculation of radiative transitions between bound states using multipole decomposition of EM field. We computed sumrules for different states of quarkonia, they showed a beautiful cancellation of decay widths of different processes taking part in the sumrule. Finally, we developed a flexible and extensible code for computing mass spectrum of Shrodinger equation, computing averages with obtained wave functions, and computing polarized matrix elements and decay widths of radiative transitions between bound states described above.

    \textbf{Keywords:} sumrules, LbL scattering, charmonium, bottomonium, quarkonium, radiative transitions
    \end{otherlanguage}
    \vspace{-1cm}
    \begin{otherlanguage}{ukrainian}
    \fontspec{Liberation Serif}[Ligatures=TeX]
    \section*{Анотація}
    \textbf{Ананьєв В.О.} ``Застосування правил сум до радіаційних переходів чармонію та ботомонію'' \\
    {\itshape Дипломна робота магістра за спеціальнітю 8.04020304 --- ядерна фізика та фізика високих енергій, спеціалізація ``квантова теорія поля''. --- Київський національний університет імені Тараса Шевченка, фізичний факультет, кафедра квантової теорії поля. --- Київ, 2018.}

    \noindent {\bfseries Науковий керівник:} Проф. д. ф-м. наук Марк Вандерхаген, Інститут ядерної фізики,
Університет ім. Йоганна Ґутенберга, Майнц.

    Правила сум --- потужний інструмент для вивчення ефектів поза нерелятивіською кварковою моделлю, що пов'язує їх з низькоенергетичними експериментальними даними. Дана робота визначила основні кроки та виявила технічні складнощі на шляху до влаштування надійного каналу між енергетичними маштабами. В даній роботі ми відтворили спектр чармонія та боттомонія на найвищому рівні і представили шляхи до котролю похибок при пошуку зв'язаних станів. Також ми розробили теоретичну базу для розрахунку радіаційних переходів між зв'язаними станами використовуючи мультипольний розклад ЕМ поля. Ми обчислили правила сум для різних станів кварконія і вони показали вражаюче скорочення ширин розпаду незалежних процесів, що дають вклад до правила. Також ми розробили гнучкий та придатний до розширення код для обчислення масового спектру рівняння Шредінгера, середніх між отриманих власних станах, поляризованих матричних елементів переходу і ширин переходу між цими станами.

    \noindent \textbf{Ключові слова:} правила сум, чармоній, боттомоній, кварконій, радіаційні переходи
    \end{otherlanguage}
    \vspace{-1cm}
    \begin{otherlanguage}{russian}
    \fontspec{Liberation Serif}[Ligatures=TeX]
    \section*{Аннотация}
    \textbf{Ананьев В.О.} ``Применение правил сумм к радиационным переходам чармония и боттомония'' \\
    {\itshape Дипломная работа магистра по специальности 8.04020304 --- ядерная физика и физика высоких энергий, специализация ``квантовая теория поля''. --- Киевский национальный университет имени Тараса Шевченко, физический факультет, кафедра квантовой теории поля. --- Киев, 2018.}

    \noindent {\bfseries Научный руководитель:} Проф. д. ф-м. наук Марк Вандерхаген, Институт ядерной физики,
Университет им. Иоганна Гутенберга.

    Правила сумм --- мощный инструмент для изучения эффектов вне нерелятивистской кварковой модели, который связывает их с низкоэнергетическими экспериментальными данными. Данная работа определила основные шаги и указала на технические сложности на пути к установлению устойчивого канала между энергетическими масштабами. В этой работе мы воспроизвели спектр чармония и боттомония с большой точностью и предложили методы контроля ошибок при поиске связанных состояний. Также мы разработали теоретическую базу для подсчета радиационных переходов между связанными состояниями, применяя мультипольное разложение ЭМ поля. Мы вычислили правила сумм для разных состояний кваркония и они показали впечатляющее сокращение ширин распада независимых процессов, дающих вклад в правило. Также мы разработали гибкий и легко расширяемый код для подсчета массового спектра уравнения Шредингера, средних между полученными собственными состояниями, поляризованных матричных элементов и ширин перехода между этими состояниями.

    \noindent \textbf{Ключевые слова:} правила сумм, чармоний, боттомоний, кварконий, радиационные переходы
    \end{otherlanguage}
\end{titlepage}
