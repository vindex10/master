\section{Tensor term averaging} \label{sec:app:tensor-term}

\begin{align}
    &H_T \propto S_T \\
    &S_T = 2(3\frac{(\vec{S}\vec{r})^2}{r^2} - \vec{S}^2) \\
    &<S_T> = \frac{1}{2J+1} \sum_{j} \left<J, j, L, S\right| S_T \left|J, j, L, S\right> \label{eq:STav}
\end{align}

As a sketch of derivation let's do the counting of Clebsch-Gordan coefficients
emerging in \cref{eq:STav}.

1. $Jj$ to $LS$ decomposition of "-in" state.
2. $Jj$ to $LS$ decomposition of "out-" state.
3. $S_T$ includes coordinate components, which are $Y_2^m$.
$\left<L, l\right| Y_2^m \left|L,l^\prime\right> \propto \left<L, l^\prime, 2, m| L, l \right>$
according to Wigner-Eckart theorem.
4. $S_T$ includes spin dependence, which is also can be expressed in terms of
spherical tensor operator, thus according to Wigner-Eckart theorem gives
$\left<S, s\right| S_{2}^m \left|S, s^\prime \right> \propto \left<S, s^\prime, 2, m| S, s\right>$

There are also $6$ summations, over all spin and momentum projections.
Hence, at firstrough look, the combination above reminds a $6j$-symbol.
Further we'd like to obtain it in a more strict fashion.

Before we start with algebra, it worths reminding a notion of tensor operators.\cref{sec:app:tensor-op}

We are now able to rewrite $S_T$ in terms of spherical tensor operators.
This action is important to decouple spin and momenta projection dependency
into Clebsch-Gordan coefficient. $S_T$ consist of the second power operators
in both coordinate and spin space, which means we are dealing
with $l=2$ and $S=2$ families. After substitution of ladder operators
instead of spin vector and collecting similar terms spherical representation
of $S_T$ looks like:

\begin{align}
    S_T = 6 \sqrt{14} \sum_{m=-2}^{2} (-1)^m S_{2}^{m} Y_{2}^{-m}
\end{align}

At this point let's factorize \cref{eq:STav} and express factors in terms
of $3j$-symbols.

\begin{align}
        &\left<S_T\right> = 6 \sqrt{14} \frac{1}{2J+1} \sum_{j} \left<JjLS\right|(-1)^m Y_2^m S_2^{-m}\left|JjLS\right> =\\
        &= \frac{6\sqrt{14}}{2J+1} \sum_{j, m_f, s_f, m_i, s_i, m} (-1)^m  \left<Jj|Lm_fSs_f\right> \left<Lm_iSs_i|Jj\right> \times \nonumber \\
        &\qquad\times \left<Lm_f\right|Y_2^m\left|Lm_i\right> \left<Ss_f\right|S_2^{-m}\left|Ss_i\right> = \\
        &= 6\sqrt{14} \sum_{j, m_f, s_f, m_i, s_i, m} (-1)^m \left\{\begin{matrix}
      L & S & J \\
      m_f & s_f & -j
    \end{matrix}\right\} \left\{\begin{matrix}
                          L & S & J \\
                          m_i & s_i & -j
                        \end{matrix}\right\} \times\nonumber \\
    &\qquad \times \left<Lm_f\right|Y_2^m\left|Lm_i\right> \left<Ss_f\right|S_2^{-m}\left|Ss_i\right> = \\
    \begin{split}
    &= 6\sqrt{14} \left<L||Y_2||L\right> \left<S||S_2||S\right> \sum_{j, m_f, s_f, m_i, s_i, m} (-1)^m \left\{\begin{matrix}
      L & S & J \\
      m_f & s_f & -j
    \end{matrix}\right\} \left\{\begin{matrix}
                          L & S & J \\
                          m_i & s_i & -j
                        \end{matrix}\right\}\\
    &\left<Lm_i2m|Lm_f\right> \left<Ss_i2(-m)|Ss_f\right> =
    \end{split}\\
    \begin{split}
        &= (-1)^{L+S} 6\sqrt{14} \sqrt{2S+1} \sqrt{2L+1} \left<L||Y_2||L\right> \left<S||S_2||S\right> \\
        &\sum_{j, m_f, s_f, m_i, s_i, m} (-1)^{m_f+s_f+m}
            \left\{\begin{matrix}
              L & S & J \\
              m_f & s_f & -j
            \end{matrix}\right\} \left\{\begin{matrix}
                                  L & S & J \\
                                  m_i & s_i & -j
                                \end{matrix}\right\} \times \nonumber \\
            &\qquad \times \left\{\begin{matrix}
                              L & 2 & L \\
                              m_i & m & -m_f
                            \end{matrix}\right\} \left\{\begin{matrix}
                                                  S & 2 & S \\
                                                  s_i & -m & -s_f
                                                \end{matrix}\right\}
       \end{split}
\end{align}

Where we have used the following convention for $3j$-symbol, normalization
of Wigen-Eckart reduced operators:

\begin{align}
    \left<LmSs|Jj\right> = (-1)^{L-S+j} \sqrt{2J+1}
                                \left\{\begin{matrix}
                                    L & S & J \\
                                    m & s & -j
                                \end{matrix}\right\}
\end{align}

\begin{align} \label{eq:WEth}
    \left<J^{(f)}j_f\right| T_D^m \left|J^{(i)} j_i\right> = \left<J^{(i)}j_iDm|J^{(f)}j_f\right> \left<J^{(f)}||T_D||J^{(i)}\right>
\end{align}

Now we should transform $3j$'s to construct a $6j$ symbol. Among with
structure of linkages in $6j$-symbol also phase factor matters. It is
easy to check, that applying momentum projections' conservation laws,
the factor can be adapted to the needed form shown below:

\begin{align}
    \left(\begin{matrix}
        L & J & S \\
        S & 2 & L
       \end{matrix}\right) = &\sum_{j,m_f,s_f,m_i,s_i,m}
                                    (-1)^{L-m_f + L-m_i + S-s_f + S-s_i + J-j + 2-m} \\ \nonumber
                                    &\left\{\begin{matrix}
                                        L & J & S \\
                                        -m_i & -j & -s_i
                                    \end{matrix}\right\}
                                    \left\{\begin{matrix}
                                        L & 2 & L \\
                                        m_i & -m & m_f
                                    \end{matrix}\right\}
                                    \left\{\begin{matrix}
                                        S & J & L \\
                                        s_f & j & -m_f
                                    \end{matrix}\right\}
                                    \left\{\begin{matrix}
                                        S & 2 & S \\
                                        -s_f & m & s_i
                                    \end{matrix}\right\}
\end{align}

Here is an intermediate result:

\begin{align} \label{eq:STavMid}
    \left<S_T\right> = (-1)^{L+S-J} 6 \sqrt{14} \sqrt{2S+1} \sqrt{2L+1} \left<L||Y_2||L\right> \left<S||S_2||S\right> \left(\begin{matrix}
            L & J & S \\
            S & 2 & L
         \end{matrix}\right)
\end{align}

\subsection{Computing reduced matrix elements}
We can use definition above (\cref{eq:WEth}), i.e. compute matrix element with
specific projections and divide it by corresponding Clebsch-Gordan coefficient.
For cross-checking we'll do it for two different configurations
of momenta projections.

\paragraph{Spin operator}
\subparagraph{m=2}
\begin{align}
        \begin{split}
            &\left<SS\right| S_2^2 \left|S (S-2)\right> = \left<SS\right| \frac{1}{2\sqrt{21}} S_{+}^2 \left|S (S-2)\right> =\\
            &= \frac{1}{\sqrt{21}} \sqrt{S(S+1) - (S-2)(S-1)} \sqrt{S(S+1) - S(S-1)} =\\
            &= \left<S (S-2) 2 2|SS\right> \left<S||S_2||S\right>
        \end{split}\\
        &\left<S||S_2||S\right> = \frac{1}{2\sqrt{21}} \frac{\sqrt{S(S+1) - (S-2)(S-1)} \sqrt{S(S+1) - S(S-1)}}{\left<S (S-2) 2 2|SS\right>} \\
        &\left<S (S-2) 2 2|SS\right> = \frac{\sqrt{6}}{\sqrt{(S+1)(2S+3)}} \\
        &\left<S||S_2||S\right> = \frac{1}{\sqrt{126}} \sqrt{S (S+1) (2S-1) (2S+3)}
\end{align}

\subparagraph{m=0 (crosscheck)}
\begin{align}
    \begin{split}
        &\left<SS\right| S_2^0 \left|SS\right> = \left<SS\right| \frac{1}{6\sqrt{14}} \left( 4 S_z^2 - \left\{S_{+}, S_{-}\right\}\right)\left|SS\right> =\\
        &= \frac{1}{6\sqrt{14}} \left( 4S^2 - \sqrt{S(S+1)-S(S-1)}^2 \right) =\\
        &= \frac{2}{6\sqrt{14}} S (2S-1) = \left<SS20|SS\right> \left<S||S_2||S\right>
    \end{split} \\
    &\left<SS20|SS\right> = \sqrt{\frac{S (2S-1)}{(S+1)(2S+3)}} \\
    &\left<S||S_2||S\right> = \frac{1}{\sqrt{126}} \sqrt{S (S+1) (2S-1) (2S+3)}
\end{align}

\paragraph{Spherical harmonics}
To compute reduced matrix element for spherical harmonics, we'll use the same
procedure. There exists known expression for integral of product
of three hamonics. We'll use it to calculate l.h.s.

\begin{align}
    \int \mathrm{d}{\Omega} \tilde{Y}_l^m \tilde{Y}_{l^\prime}^{m^\prime} (\tilde{Y}_J^j)^{\star} = \sqrt{\frac{(2l+1)(2l^\prime+1)}{4\pi(2J+1)}} \left<l0l^\prime0|J0\right> \left<lml^\prime m^\prime|Jj\right>
\end{align}

\begin{align}
    \begin{split}
        &\left<Ll_f\right| Y_2^m \left|L l_i\right> = 2\sqrt{\frac{\pi}{5}} \int \mathrm{d}\Omega (\tilde{Y}_L^{l_f})^\star Y_2^m \tilde{Y}_L^{l_i} =\\
        &= \left<L020|L0\right> \left<Ll_i2m|Ll_f\right> = \left<L||Y_2||L\right> \left<Ll_i2m|Ll_f\right>
    \end{split}\\
    &\left<L020|L0\right> = -\sqrt{\frac{(L+1)L}{(2L-1)(2L+3)}}\\
    &\left<L||Y_2||L\right> = -\sqrt{\frac{(L+1)L}{(2L-1)(2L+3)}}
\end{align}

After substituting reduced matrix elements into \cref{eq:STavMid} and collecting everything together:

\begin{align}
    &\left<S_T\right> = 2 (-1)^{L+S-J+1} \sqrt{\frac{L(L+1)(2L+1)}{(2L-1)(2L+3)}} \times \nonumber \\
    &\qquad\times \sqrt{S(S+1)(2S+1)(2S+3)(2S-1)} \left(\begin{matrix}
                                              L & J & S \\
                                              S & 2 & L
                                           \end{matrix}\right)
\end{align}
