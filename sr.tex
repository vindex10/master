\chapter{Sumrules}

At this point we have obtained spectrum and transition coefficients for charmonium and bottomonium bound states. We are able now to compute sumrules. We present results in a table where we list contributions of each transition. There are three columns: results based on PDG data, results based on Deng et al. and results of our computations. As one could notice, not all decay widths we need are provided by PDG and Deng. We fill blank spaces ($nan$'s) with our results when required.

Let's start with $\psi(1S)$ sumrules.

From values of totals~(\cref{tab:sr.psi_1S.c-scr}) it follows that absolute values of decay widths computed by us are not precise enough, at the same time their fractions reproduce $r^{(0,2)}$ quite well. One can trace the cancellation of large numbers in $PDG$ data. Moreover, it can be seen that after taking into account higher multipoles the picture for $PDG$ and $Deng$ improves. It can serve as an evidence of the well computed transition coefficients.

\begin{table}[H]
    {\caption{Sumrule computed for $\psi(1S)$ with multipoles up to J=8 measured in $\mu b$. Experimental error $\sigma \approx 6.7 \mu b$ mostly comes from 1P decays \label{tab:sr.psi_1S.c-scr}}}
    \begin{tabular}{|l|l|c|c|c|c|c|c|}%
\hline%
&&PDG&$SR-PDG$&Deng&$SR-Deng$&$\Gamma$&$SR-\Gamma$\\%
In&Out&&&&&&\\%
\hline%
$\psi(1S)$&$\eta_{c}(1S)$&1.58&-13.34&2.44&-20.60&2.79&-23.58\\%
$\chi_{c0}(1P)$&$\psi(1S)$&151.20&-20.88&179.00&-24.72&141.78&-19.58\\%
$\chi_{c1}(1P)$&$\psi(1S)$&288.12&-32.02&319.00&-35.45&310.01&-34.45\\%
$\chi_{c2}(1P)$&$\psi(1S)$&374.30&34.11&292.00&26.61&510.43&46.52\\%
\hline%
\hline%
\multicolumn{2}{|c|}{Subtotal}&\multicolumn{2}{|r|}{$-32.12 (-4.79\sigma)$}&\multicolumn{2}{|r|}{$-54.16 (-8.08\sigma)$}&\multicolumn{2}{|r|}{$-31.10 (-4.64\sigma)$}\\%
\hline%
\hline%
$\eta_{c}(2S)$&$\psi(1S)$&nan&-0.09&2.29&-0.07&2.91&-0.09\\%
$\chi_{c0}(2P)$&$\psi(1S)$&nan&-0.02&2.30&-0.03&1.89&-0.02\\%
$\chi_{c1}(2P)$&$\psi(1S)$&nan&-1.58&88.00&-1.86&74.43&-1.58\\%
$\chi_{c2}(2P)$&$\psi(1S)$&nan&3.36&93.00&1.32&237.32&3.36\\%
\hline%
\hline%
\multicolumn{2}{|c|}{Subtotal}&\multicolumn{2}{|r|}{$1.67~(0.25\sigma)$}&\multicolumn{2}{|r|}{$-0.64~(-0.10\sigma)$}&\multicolumn{2}{|r|}{$1.67~(0.25\sigma)$}\\%
\hline%
\hline%
\multicolumn{2}{|c|}{Total}&\multicolumn{2}{|r|}{$-30.45~(-4.54\sigma)$}&\multicolumn{2}{|r|}{$-54.81~(-8.18\sigma)$}&\multicolumn{2}{|r|}{$-29.42~(-4.39\sigma)$}\\%
\hline%
\end{tabular}
\end{table}

\begin{table}[H]
    {\caption{Sumrule computed for $\psi(1S)$ in E1 approximation, measured in $\mu b$. Experimental error $\sigma \approx 6.7 \mu b$ mostly comes from 1P decays \label{tab:srE1.psi_1S.c-scr}}}
    \begin{tabular}{|l|l|c|c|c|c|c|c|}%
\hline%
&&PDG&$SR-PDG$&Deng&$SR-Deng$&E1-$\Gamma$&$SR-\Gamma$\\%
In&Out&&&&&&\\%
\hline%
$\chi_{c0}(1P)$&$\psi(1S)$&151.20&-20.88&179.00&-24.72&141.78&-19.58\\%
$\chi_{c1}(1P)$&$\psi(1S)$&288.12&-28.13&319.00&-31.15&308.53&-30.12\\%
$\chi_{c2}(1P)$&$\psi(1S)$&374.30&45.35&292.00&35.38&506.40&61.36\\%
\hline%
\hline%
\multicolumn{2}{|c|}{Subtotal}&\multicolumn{2}{|r|}{$-3.66 (-0.55\sigma)$}&\multicolumn{2}{|r|}{$-20.49 (-3.06\sigma)$}&\multicolumn{2}{|r|}{$11.65 (1.74\sigma)$}\\%
\hline%
\hline%
$\chi_{c0}(2P)$&$\psi(1S)$&nan&-0.02&2.30&-0.03&1.89&-0.02\\%
$\chi_{c1}(2P)$&$\psi(1S)$&nan&-1.25&88.00&-1.50&73.29&-1.25\\%
$\chi_{c2}(2P)$&$\psi(1S)$&nan&5.46&93.00&2.18&232.70&5.46\\%
\hline%
\hline%
\multicolumn{2}{|c|}{Subtotal}&\multicolumn{2}{|r|}{$4.19~(0.63\sigma)$}&\multicolumn{2}{|r|}{$0.66~(0.10\sigma)$}&\multicolumn{2}{|r|}{$4.19~(0.63\sigma)$}\\%
\hline%
\hline%
\multicolumn{2}{|c|}{Total}&\multicolumn{2}{|r|}{$0.53~(0.08\sigma)$}&\multicolumn{2}{|r|}{$-19.83~(-2.96\sigma)$}&\multicolumn{2}{|r|}{$15.84~(2.36\sigma)$}\\%
\hline%
\end{tabular}
\end{table}

\begin{table}[H]
    {\caption{Sumrule computed for $\psi(1S)$ in ELW approximation, measured in $\mu b$. Experimental error $\sigma \approx 6.7 \mu b$ mostly comes from 1P decays \label{tab:srELW.psi_1S.c-scr}}}
    \begin{tabular}{|l|l|c|c|c|c|c|c|}%
\hline%
&&PDG&$SR-PDG$&Deng&$SR-Deng$&ELW-$\Gamma$&$SR-\Gamma$\\%
In&Out&&&&&&\\%
\hline%
$\chi_{c0}(1P)$&$\psi(1S)$&133.35&-18.39&179.00&-24.68&186.99&-25.78\\%
$\chi_{c1}(1P)$&$\psi(1S)$&284.76&-27.76&319.00&-31.10&411.10&-40.08\\%
$\chi_{c2}(1P)$&$\psi(1S)$&370.56&44.97&292.00&35.44&480.79&58.35\\%
\hline%
\hline%
\multicolumn{2}{|c|}{Subtotal}&\multicolumn{2}{|r|}{$-1.17 (-0.18\sigma)$}&\multicolumn{2}{|r|}{$-20.34 (-3.04\sigma)$}&\multicolumn{2}{|r|}{$-7.51 (-1.12\sigma)$}\\%
\hline%
\hline%
$\chi_{c0}(2P)$&$\psi(1S)$&nan&-0.03&2.30&-0.02&3.15&-0.03\\%
$\chi_{c1}(2P)$&$\psi(1S)$&nan&-0.88&88.00&-1.30&60.07&-0.88\\%
$\chi_{c2}(2P)$&$\psi(1S)$&nan&3.20&93.00&2.18&136.18&3.20\\%
\hline%
\hline%
\multicolumn{2}{|c|}{Subtotal}&\multicolumn{2}{|r|}{$2.28~(0.34\sigma)$}&\multicolumn{2}{|r|}{$0.87~(0.13\sigma)$}&\multicolumn{2}{|r|}{$2.28~(0.34\sigma)$}\\%
\hline%
\hline%
\multicolumn{2}{|c|}{Total}&\multicolumn{2}{|r|}{$1.11~(0.17\sigma)$}&\multicolumn{2}{|r|}{$-19.47~(-2.91\sigma)$}&\multicolumn{2}{|r|}{$-5.22~(-0.78\sigma)$}\\%
\hline%
\end{tabular}
\end{table}

For comparison of different potentials, we present sumrules for $\psi(1S)$ in case of linear potential. At least for low-lying states there is no significant difference between linear and screening potential when comparing sumrules.

\begin{table}[H]
    {\caption{Sumrule computed for $\psi(1S)$ in linear potential with multipoles up to J=8 measured in $\mu b$. Experimental error $\sigma \approx 6.7 \mu b$ mostly comes from 1P decays \label{tab:sr.psi_1S.c-lin}}}
    \begin{tabular}{|l|l|c|c|c|c|c|c|}%
\hline%
&&PDG&$SR-PDG$&Deng&$SR-Deng$&$\Gamma$&$SR-\Gamma$\\%
In&Out&&&&&&\\%
\hline%
$\psi(1S)$&$\eta_{c}(1S)$&1.58&-13.34&2.39&-20.18&2.57&-21.74\\%
$\chi_{c0}(1P)$&$\psi(1S)$&133.35&-18.39&172.00&-23.72&140.33&-19.35\\%
$\chi_{c1}(1P)$&$\psi(1S)$&284.76&-31.44&306.00&-33.78&310.50&-34.28\\%
$\chi_{c2}(1P)$&$\psi(1S)$&370.56&34.44&284.00&26.39&476.53&44.29\\%
\hline%
\hline%
\multicolumn{2}{|c|}{Subtotal}&\multicolumn{2}{|r|}{$-28.72 (-4.29\sigma)$}&\multicolumn{2}{|r|}{$-51.29 (-7.65\sigma)$}&\multicolumn{2}{|r|}{$-31.08 (-4.64\sigma)$}\\%
\hline%
\hline%
$\eta_{c}(2S)$&$\psi(1S)$&nan&-0.09&2.64&-0.08&3.03&-0.09\\%
$\chi_{c0}(2P)$&$\psi(1S)$&nan&-0.03&6.10&-0.06&2.89&-0.03\\%
$\chi_{c1}(2P)$&$\psi(1S)$&nan&-1.55&81.00&-1.38&90.94&-1.55\\%
$\chi_{c2}(2P)$&$\psi(1S)$&nan&3.86&93.00&1.33&270.25&3.86\\%
\hline%
\hline%
\multicolumn{2}{|c|}{Subtotal}&\multicolumn{2}{|r|}{$2.19~(0.33\sigma)$}&\multicolumn{2}{|r|}{$-0.19~(-0.03\sigma)$}&\multicolumn{2}{|r|}{$2.19~(0.33\sigma)$}\\%
\hline%
\hline%
\multicolumn{2}{|c|}{Total}&\multicolumn{2}{|r|}{$-26.53~(-3.96\sigma)$}&\multicolumn{2}{|r|}{$-51.48~(-7.68\sigma)$}&\multicolumn{2}{|r|}{$-28.89~(-4.31\sigma)$}\\%
\hline%
\end{tabular}
\end{table}

\begin{table}[H]
    {\caption{Sumrule computed for $\psi(1S)$ in linear potential in E1 approximation, measured in $\mu b$. Experimental error $\sigma \approx 6.7 \mu b$ mostly comes from 1P decays \label{tab:srE1.psi_1S.c-lin}}}
    \begin{tabular}{|l|l|c|c|c|c|c|c|}%
\hline%
&&PDG&$SR-PDG$&Deng&$SR-Deng$&E1-$\Gamma$&$SR-\Gamma$\\%
In&Out&&&&&&\\%
\hline%
$\chi_{c0}(1P)$&$\psi(1S)$&151.20&-20.88&172.00&-23.76&138.28&-19.10\\%
$\chi_{c1}(1P)$&$\psi(1S)$&288.12&-28.13&306.00&-29.88&299.88&-29.28\\%
$\chi_{c2}(1P)$&$\psi(1S)$&374.30&45.35&284.00&34.41&486.34&58.93\\%
\hline%
\hline%
\multicolumn{2}{|c|}{Subtotal}&\multicolumn{2}{|r|}{$-3.66 (-0.55\sigma)$}&\multicolumn{2}{|r|}{$-19.22 (-2.87\sigma)$}&\multicolumn{2}{|r|}{$10.55 (1.57\sigma)$}\\%
\hline%
\hline%
$\chi_{c0}(2P)$&$\psi(1S)$&nan&-0.03&6.10&-0.07&2.66&-0.03\\%
$\chi_{c1}(2P)$&$\psi(1S)$&nan&-1.16&81.00&-1.38&68.19&-1.16\\%
$\chi_{c2}(2P)$&$\psi(1S)$&nan&5.33&93.00&2.18&227.23&5.33\\%
\hline%
\hline%
\multicolumn{2}{|c|}{Subtotal}&\multicolumn{2}{|r|}{$4.14~(0.62\sigma)$}&\multicolumn{2}{|r|}{$0.73~(0.11\sigma)$}&\multicolumn{2}{|r|}{$4.14~(0.62\sigma)$}\\%
\hline%
\hline%
\multicolumn{2}{|c|}{Total}&\multicolumn{2}{|r|}{$0.48~(0.07\sigma)$}&\multicolumn{2}{|r|}{$-18.49~(-2.76\sigma)$}&\multicolumn{2}{|r|}{$14.69~(2.19\sigma)$}\\%
\hline%
\end{tabular}
\end{table}

\begin{table}[H]
    {\caption{Sumrule computed for $\psi(1S)$ in linear potential in ELW approximation, measured in $\mu b$. Experimental error $\sigma \approx 6.7 \mu b$ mostly comes from 1P decays \label{tab:srELW.psi_1S.c-lin}}}
    \begin{tabular}{|l|l|c|c|c|c|c|c|}%
\hline%
&&PDG&$SR-PDG$&Deng&$SR-Deng$&ELW-$\Gamma$&$SR-\Gamma$\\%
In&Out&&&&&&\\%
\hline%
$\chi_{c0}(1P)$&$\psi(1S)$&133.35&-18.39&172.00&-23.72&179.53&-24.75\\%
$\chi_{c1}(1P)$&$\psi(1S)$&284.76&-27.76&306.00&-29.83&380.00&-37.04\\%
$\chi_{c2}(1P)$&$\psi(1S)$&370.56&44.97&284.00&34.47&458.05&55.59\\%
\hline%
\hline%
\multicolumn{2}{|c|}{Subtotal}&\multicolumn{2}{|r|}{$-1.17 (-0.18\sigma)$}&\multicolumn{2}{|r|}{$-19.08 (-2.85\sigma)$}&\multicolumn{2}{|r|}{$-6.21 (-0.93\sigma)$}\\%
\hline%
\hline%
$\chi_{c0}(2P)$&$\psi(1S)$&nan&-0.02&6.10&-0.06&1.71&-0.02\\%
$\chi_{c1}(2P)$&$\psi(1S)$&nan&-0.79&81.00&-1.10&57.99&-0.79\\%
$\chi_{c2}(2P)$&$\psi(1S)$&nan&3.43&93.00&2.18&146.17&3.43\\%
\hline%
\hline%
\multicolumn{2}{|c|}{Subtotal}&\multicolumn{2}{|r|}{$2.63~(0.39\sigma)$}&\multicolumn{2}{|r|}{$1.02~(0.15\sigma)$}&\multicolumn{2}{|r|}{$2.63~(0.39\sigma)$}\\%
\hline%
\hline%
\multicolumn{2}{|c|}{Total}&\multicolumn{2}{|r|}{$1.45~(0.22\sigma)$}&\multicolumn{2}{|r|}{$-18.06~(-2.70\sigma)$}&\multicolumn{2}{|r|}{$-3.58~(-0.53\sigma)$}\\%
\hline%
\end{tabular}
\end{table}

Following the same setting as for decay widths, we provide results for sumurules computed for bottomonium $\Upsilon(1S)$

\begin{table}[H]
    \caption{Sumrule computed for $\Upsilon(1S)$ in screened potential with multipoles up to J=8 measured in $\mu b$. Error $\sigma \approx 0.22 \mu b$ mostly comes from comparison to Deng et al. \label{tab:sr.yps_1S.b-scr}}
    \begin{tabular}{|l|l|c|c|c|c|c|c|}%
\hline%
&&PDG&$SR-PDG$&Deng&$SR-Deng$&$\Gamma$&$SR-\Gamma$\\%
In&Out&&&&&&\\%
\hline%
$\chi_{b0}(1P)$&$\Upsilon(1S)$&nan&-1.68&27.50&-1.77&26.09&-1.68\\%
$\chi_{b1}(1P)$&$\Upsilon(1S)$&nan&-2.43&31.90&-2.42&32.05&-2.43\\%
$\chi_{b2}(1P)$&$\Upsilon(1S)$&nan&3.83&31.80&3.55&34.37&3.83\\%
\hline%
\hline%
\multicolumn{2}{|c|}{Subtotal}&\multicolumn{2}{|r|}{$-0.28 (-23.22\sigma)$}&\multicolumn{2}{|r|}{$-0.64 (-53.67\sigma)$}&\multicolumn{2}{|r|}{$-0.28 (-23.22\sigma)$}\\%
\hline%
\hline%
$\chi_{b0}(2P)$&$\Upsilon(1S)$&nan&-0.07&5.54&-0.05&7.07&-0.07\\%
$\chi_{b1}(2P)$&$\Upsilon(1S)$&nan&-0.18&10.80&-0.14&13.97&-0.18\\%
$\chi_{b2}(2P)$&$\Upsilon(1S)$&nan&0.37&12.50&0.26&18.09&0.37\\%
\hline%
\hline%
\multicolumn{2}{|c|}{Subtotal}&\multicolumn{2}{|r|}{$0.12 (10.31\sigma)$}&\multicolumn{2}{|r|}{$0.06 (5.39\sigma)$}&\multicolumn{2}{|r|}{$0.12 (10.31\sigma)$}\\%
\hline%
\hline%
$\chi_{b0}(3P)$&$\Upsilon(1S)$&nan&-0.01&1.87&-0.01&3.24&-0.01\\%
$\chi_{b1}(3P)$&$\Upsilon(1S)$&nan&-0.05&6.41&-0.04&8.84&-0.05\\%
$\chi_{b2}(3P)$&$\Upsilon(1S)$&nan&0.12&8.17&0.08&12.71&0.12\\%
\hline%
\hline%
\multicolumn{2}{|c|}{Subtotal}&\multicolumn{2}{|r|}{$0.05~(4.41\sigma)$}&\multicolumn{2}{|r|}{$0.03~(2.55\sigma)$}&\multicolumn{2}{|r|}{$0.05~(4.41\sigma)$}\\%
\hline%
\hline%
\multicolumn{2}{|c|}{Total}&\multicolumn{2}{|r|}{$-0.10~(-8.51\sigma)$}&\multicolumn{2}{|r|}{$-0.55~(-45.73\sigma)$}&\multicolumn{2}{|r|}{$-0.10~(-8.51\sigma)$}\\%
\hline%
\end{tabular}
\end{table}

\begin{table}[H]
    \caption{Sumrule computed for $\Upsilon(1S)$ in screened potential in E1 approximation, measured in $\mu b$. Error $\sigma \approx 0.19 \mu b$ mostly comes from comparison to Deng et al. \label{tab:srE1.yps_1S.b-scr}}
    \begin{tabular}{|l|l|c|c|c|c|c|c|}%
\hline%
&&PDG&$SR-PDG$&Deng&$SR-Deng$&E1-$\Gamma$&$SR-\Gamma$\\%
In&Out&&&&&&\\%
\hline%
$\chi_{b0}(1P)$&$\Upsilon(1S)$&nan&-1.75&27.50&-1.77&27.19&-1.75\\%
$\chi_{b1}(1P)$&$\Upsilon(1S)$&nan&-2.57&31.90&-2.42&33.83&-2.57\\%
$\chi_{b2}(1P)$&$\Upsilon(1S)$&nan&4.07&31.80&3.55&36.53&4.07\\%
\hline%
\hline%
\multicolumn{2}{|c|}{Subtotal}&\multicolumn{2}{|r|}{$-0.24 (-1.31\sigma)$}&\multicolumn{2}{|r|}{$-0.64 (-3.45\sigma)$}&\multicolumn{2}{|r|}{$-0.24 (-1.31\sigma)$}\\%
\hline%
\hline%
$\chi_{b0}(2P)$&$\Upsilon(1S)$&nan&-0.05&5.54&-0.05&4.92&-0.05\\%
$\chi_{b1}(2P)$&$\Upsilon(1S)$&nan&-0.15&10.80&-0.14&11.33&-0.15\\%
$\chi_{b2}(2P)$&$\Upsilon(1S)$&nan&0.32&12.50&0.26&15.44&0.32\\%
\hline%
\hline%
\multicolumn{2}{|c|}{Subtotal}&\multicolumn{2}{|r|}{$0.12 (0.66\sigma)$}&\multicolumn{2}{|r|}{$0.06 (0.35\sigma)$}&\multicolumn{2}{|r|}{$0.12 (0.66\sigma)$}\\%
\hline%
\hline%
$\chi_{b0}(3P)$&$\Upsilon(1S)$&nan&-0.01&1.87&-0.01&1.68&-0.01\\%
$\chi_{b1}(3P)$&$\Upsilon(1S)$&nan&-0.04&6.41&-0.04&6.21&-0.04\\%
$\chi_{b2}(3P)$&$\Upsilon(1S)$&nan&0.09&8.17&0.08&9.71&0.09\\%
\hline%
\hline%
\multicolumn{2}{|c|}{Subtotal}&\multicolumn{2}{|r|}{$0.05~(0.25\sigma)$}&\multicolumn{2}{|r|}{$0.03~(0.16\sigma)$}&\multicolumn{2}{|r|}{$0.05~(0.25\sigma)$}\\%
\hline%
\hline%
\multicolumn{2}{|c|}{Total}&\multicolumn{2}{|r|}{$-0.07~(-0.39\sigma)$}&\multicolumn{2}{|r|}{$-0.55~(-2.94\sigma)$}&\multicolumn{2}{|r|}{$-0.07~(-0.39\sigma)$}\\%
\hline%
\end{tabular}
\end{table}

\begin{table}[H]
    \caption{Sumrule computed for $\Upsilon(1S)$ in screened potential in ELW approximation, measured in $\mu b$. Error $\sigma \approx 0.24 \mu b$ mostly comes from comparison to Deng et al. \label{tab:srELW.yps_1S.b-scr}}
    \begin{tabular}{|l|l|c|c|c|c|c|c|}%
\hline%
&&PDG&$SR-PDG$&Deng&$SR-Deng$&ELW-$\Gamma$&$SR-\Gamma$\\%
In&Out&&&&&&\\%
\hline%
$\chi_{b0}(1P)$&$\Upsilon(1S)$&nan&-1.79&27.50&-1.77&27.85&-1.79\\%
$\chi_{b1}(1P)$&$\Upsilon(1S)$&nan&-2.65&31.90&-2.42&34.90&-2.65\\%
$\chi_{b2}(1P)$&$\Upsilon(1S)$&nan&4.22&31.80&3.55&37.83&4.22\\%
\hline%
\hline%
\multicolumn{2}{|c|}{Subtotal}&\multicolumn{2}{|r|}{$-0.22 (-0.91\sigma)$}&\multicolumn{2}{|r|}{$-0.64 (-2.63\sigma)$}&\multicolumn{2}{|r|}{$-0.22 (-0.91\sigma)$}\\%
\hline%
\hline%
$\chi_{b0}(2P)$&$\Upsilon(1S)$&nan&-0.04&5.54&-0.05&3.79&-0.04\\%
$\chi_{b1}(2P)$&$\Upsilon(1S)$&nan&-0.13&10.80&-0.14&9.79&-0.13\\%
$\chi_{b2}(2P)$&$\Upsilon(1S)$&nan&0.28&12.50&0.26&13.84&0.28\\%
\hline%
\hline%
\multicolumn{2}{|c|}{Subtotal}&\multicolumn{2}{|r|}{$0.12 (0.50\sigma)$}&\multicolumn{2}{|r|}{$0.06 (0.26\sigma)$}&\multicolumn{2}{|r|}{$0.12 (0.50\sigma)$}\\%
\hline%
\hline%
$\chi_{b0}(3P)$&$\Upsilon(1S)$&nan&-0.00&1.87&-0.01&1.01&-0.00\\%
$\chi_{b1}(3P)$&$\Upsilon(1S)$&nan&-0.03&6.41&-0.04&4.87&-0.03\\%
$\chi_{b2}(3P)$&$\Upsilon(1S)$&nan&0.07&8.17&0.08&8.10&0.07\\%
\hline%
\hline%
\multicolumn{2}{|c|}{Subtotal}&\multicolumn{2}{|r|}{$0.04~(0.17\sigma)$}&\multicolumn{2}{|r|}{$0.03~(0.13\sigma)$}&\multicolumn{2}{|r|}{$0.04~(0.17\sigma)$}\\%
\hline%
\hline%
\multicolumn{2}{|c|}{Total}&\multicolumn{2}{|r|}{$-0.06~(-0.24\sigma)$}&\multicolumn{2}{|r|}{$-0.55~(-2.24\sigma)$}&\multicolumn{2}{|r|}{$-0.06~(-0.24\sigma)$}\\%
\hline%
\end{tabular}
\end{table}

By analysing these three cases we came to the opinion that Deng et al. reproduced absolute values of decay widths better then we, because sumrules cancel better for results provided by Deng et al for both charmonium and bottomonium. Nevertheless, our coefficients work quite well.

Let's follow to $\psi(2S)$ and $\Upsilon(2S)$ states. We omit linear potential because results are similar to the screened one. 

\begin{table}[H]
    {\caption{Sumrule computed for $\psi(2S)$ with multipoles up to J=8 measured in $\mu b$. Experimental error $\sigma \approx 16.6 \mu b$ mostly comes from $\eta_c(2S)$ decay \label{tab:sr.psi_2S.c-scr}}}
    \begin{tabular}{l|l|c|c|c|c|c|c}
\toprule
                &            &   PDG & $SR-PDG$ &  Deng & $SR-Deng$ &  $\Gamma$ & $SR-\Gamma$ \\
\textbf{In} & \textbf{Out} &       &          &       &           &           &             \\
\midrule
\textbf{$\psi(2S)$} & \textbf{$\eta_{c}(2S)$} & 0.207 &    -23.5 &  0.19 &     -21.5 &     0.148 &       -16.8 \\
                & \textbf{$\eta_{c}(1S)$} &  1.01 &  -0.0453 &   7.8 &    -0.351 &      10.3 &      -0.462 \\
                & \textbf{$\chi_{c2}(1P)$} &    27 &     68.8 &    46 &       117 &      42.8 &         109 \\
                & \textbf{$\chi_{c1}(1P)$} &  28.3 &    -34.5 &    45 &     -54.9 &      34.9 &       -42.6 \\
                & \textbf{$\chi_{c0}(1P)$} &  29.6 &    -19.2 &    22 &     -14.3 &      20.2 &       -13.1 \\
\textbf{$\chi_{c2}(2P)$} & \textbf{$\psi(2S)$} &   nan &      141 &   150 &       133 &       159 &         141 \\
\textbf{$\chi_{c1}(2P)$} & \textbf{$\psi(2S)$} &   nan &    -92.6 &   155 &     -75.7 &       189 &       -92.6 \\
\textbf{$\chi_{c0}(2P)$} & \textbf{$\psi(2S)$} &   nan &    -38.9 &    99 &     -33.3 &       116 &       -38.9 \\
\textbf{$\chi_{c2}(3P)$} & \textbf{$\psi(2S)$} &   nan &     15.7 &    76 &      7.66 &       156 &        15.7 \\
\textbf{$\chi_{c1}(3P)$} & \textbf{$\psi(2S)$} &   nan &    -8.53 &    74 &     -3.53 &       179 &       -8.53 \\
\textbf{$\chi_{c0}(3P)$} & \textbf{$\psi(2S)$} &   nan &    -3.38 &   9.1 &    -0.425 &      72.4 &       -3.38 \\
\bottomrule
\end{tabular}

\end{table}

\begin{table}[H]
    {\caption{Sumrule computed for $\psi(2S)$ in E1 approximation, measured in $\mu b$. Experimental error $\sigma \approx 16.6 \mu b$ mostly comes from $\eta_c(2S)$ decay \label{tab:srE1.psi_2S.c-scr}}}
    \begin{tabular}{|l|l|c|c|c|c|c|c|}%
\hline%
&&PDG&$SR-PDG$&Deng&$SR-Deng$&E1-$\Gamma$&$SR-\Gamma$\\%
In&Out&&&&&&\\%
\hline%
$\psi(2S)$&$\eta_{c}(1S)$&1.01&-0.05&7.80&-0.35&10.26&-0.46\\%
$\psi(2S)$&$\chi_{c0}(1P)$&29.57&-19.17&22.00&-14.26&28.08&-18.20\\%
$\psi(2S)$&$\chi_{c1}(1P)$&28.27&-32.68&45.00&-52.02&38.76&-44.80\\%
$\psi(2S)$&$\chi_{c2}(1P)$&26.97&74.63&46.00&127.31&42.37&117.27\\%
\hline%
\hline%
\multicolumn{2}{|c|}{Subtotal}&\multicolumn{2}{|r|}{$22.74 (1.37\sigma)$}&\multicolumn{2}{|r|}{$60.68 (3.66\sigma)$}&\multicolumn{2}{|r|}{$53.81 (3.24\sigma)$}\\%
\hline%
\hline%
$\psi(2S)$&$\eta_{c}(2S)$&0.21&-23.45&0.19&-21.51&0.15&-16.79\\%
$\chi_{c0}(2P)$&$\psi(2S)$&nan&-37.58&99.00&-33.34&111.59&-37.58\\%
$\chi_{c1}(2P)$&$\psi(2S)$&nan&-110.98&155.00&-82.51&208.48&-110.98\\%
$\chi_{c2}(2P)$&$\psi(2S)$&nan&167.51&150.00&113.05&222.26&167.51\\%
\hline%
\hline%
\multicolumn{2}{|c|}{Subtotal}&\multicolumn{2}{|r|}{$-4.51 (-0.27\sigma)$}&\multicolumn{2}{|r|}{$-24.31 (-1.46\sigma)$}&\multicolumn{2}{|r|}{$2.15 (0.13\sigma)$}\\%
\hline%
\hline%
$\chi_{c0}(3P)$&$\psi(2S)$&nan&-0.47&9.10&-0.42&9.98&-0.47\\%
$\chi_{c1}(3P)$&$\psi(2S)$&nan&-4.50&74.00&-3.97&83.84&-4.50\\%
$\chi_{c2}(3P)$&$\psi(2S)$&nan&9.68&76.00&6.14&119.86&9.68\\%
\hline%
\hline%
\multicolumn{2}{|c|}{Subtotal}&\multicolumn{2}{|r|}{$4.72~(0.28\sigma)$}&\multicolumn{2}{|r|}{$1.75~(0.11\sigma)$}&\multicolumn{2}{|r|}{$4.72~(0.28\sigma)$}\\%
\hline%
\hline%
\multicolumn{2}{|c|}{Total}&\multicolumn{2}{|r|}{$22.96~(1.38\sigma)$}&\multicolumn{2}{|r|}{$38.12~(2.30\sigma)$}&\multicolumn{2}{|r|}{$60.68~(3.66\sigma)$}\\%
\hline%
\end{tabular}
\end{table}

\begin{table}[H]
    {\caption{Sumrule computed for $\psi(2S)$ in ELW approximation, measured in $\mu b$. Experimental error $\sigma \approx 16.6 \mu b$ mostly comes from $\eta_c(2S)$ decay \label{tab:srELW.psi_2S.c-scr}}}
    \begin{tabular}{|l|l|c|c|c|c|c|c|}%
\hline%
&&PDG&$SR-PDG$&Deng&$SR-Deng$&ELW-$\Gamma$&$SR-\Gamma$\\%
In&Out&&&&&&\\%
\hline%
$\psi(2S)$&$\chi_{c0}(1P)$&29.57&-19.17&22.00&-14.26&30.31&-19.65\\%
$\psi(2S)$&$\chi_{c1}(1P)$&28.27&-32.68&45.00&-52.02&39.90&-46.13\\%
$\psi(2S)$&$\chi_{c2}(1P)$&26.97&74.63&46.00&127.31&43.19&119.53\\%
\hline%
\hline%
\multicolumn{2}{|c|}{Subtotal}&\multicolumn{2}{|r|}{$22.79 (1.37\sigma)$}&\multicolumn{2}{|r|}{$61.03 (3.68\sigma)$}&\multicolumn{2}{|r|}{$53.75 (3.24\sigma)$}\\%
\hline%
\hline%
$\chi_{c0}(2P)$&$\psi(2S)$&nan&-39.37&99.00&-33.34&116.90&-39.37\\%
$\chi_{c1}(2P)$&$\psi(2S)$&nan&-122.71&155.00&-82.51&230.52&-122.71\\%
$\chi_{c2}(2P)$&$\psi(2S)$&nan&191.06&150.00&113.05&253.51&191.06\\%
\hline%
\hline%
\multicolumn{2}{|c|}{Subtotal}&\multicolumn{2}{|r|}{$28.98 (1.75\sigma)$}&\multicolumn{2}{|r|}{$-2.81 (-0.17\sigma)$}&\multicolumn{2}{|r|}{$28.98 (1.75\sigma)$}\\%
\hline%
\hline%
$\chi_{c0}(3P)$&$\psi(2S)$&nan&-0.00&9.10&-0.42&0.04&-0.00\\%
$\chi_{c1}(3P)$&$\psi(2S)$&nan&-2.23&74.00&-3.97&41.67&-2.23\\%
$\chi_{c2}(3P)$&$\psi(2S)$&nan&6.30&76.00&6.14&77.99&6.30\\%
\hline%
\hline%
\multicolumn{2}{|c|}{Subtotal}&\multicolumn{2}{|r|}{$4.06~(0.24\sigma)$}&\multicolumn{2}{|r|}{$1.75~(0.11\sigma)$}&\multicolumn{2}{|r|}{$4.06~(0.24\sigma)$}\\%
\hline%
\hline%
\multicolumn{2}{|c|}{Total}&\multicolumn{2}{|r|}{$55.83~(3.36\sigma)$}&\multicolumn{2}{|r|}{$59.98~(3.61\sigma)$}&\multicolumn{2}{|r|}{$86.79~(5.23\sigma)$}\\%
\hline%
\end{tabular}
\end{table}

\begin{table}[H]
    \caption{Sumrule computed for $\Upsilon(2S)$ in screened potential with multipoles up to J=8 measured in $\mu b$. Error $\sigma \approx 1.85 \mu b$ comes from comparison to Deng et al \label{tab:sr.yps_2S.b-scr}}
    \begin{tabular}{|l|l|c|c|c|c|c|c|}%
\hline%
&&PDG&$SR-PDG$&Deng&$SR-Deng$&$\Gamma$&$SR-\Gamma$\\%
In&Out&&&&&&\\%
\hline%
$\Upsilon(2S)$&$\chi_{b0}(1P)$&1.22&-3.26&1.09&-2.92&0.93&-2.49\\%
$\Upsilon(2S)$&$\chi_{b1}(1P)$&2.21&-5.84&2.17&-5.74&1.44&-3.82\\%
$\Upsilon(2S)$&$\chi_{b2}(1P)$&2.29&9.97&2.62&11.43&1.57&6.86\\%
\hline%
\hline%
\multicolumn{2}{|c|}{Subtotal}&\multicolumn{2}{|r|}{$0.88 (0.48\sigma)$}&\multicolumn{2}{|r|}{$2.77 (1.50\sigma)$}&\multicolumn{2}{|r|}{$0.55 (0.30\sigma)$}\\%
\hline%
\hline%
$\chi_{b0}(2P)$&$\Upsilon(2S)$&nan&-5.33&14.40&-6.16&12.46&-5.33\\%
$\chi_{b1}(2P)$&$\Upsilon(2S)$&nan&-7.63&15.30&-7.49&15.60&-7.63\\%
$\chi_{b2}(2P)$&$\Upsilon(2S)$&nan&11.11&15.30&9.77&17.40&11.11\\%
\hline%
\hline%
\multicolumn{2}{|c|}{Subtotal}&\multicolumn{2}{|r|}{$-1.85 (-1.00\sigma)$}&\multicolumn{2}{|r|}{$-3.88 (-2.10\sigma)$}&\multicolumn{2}{|r|}{$-1.85 (-1.00\sigma)$}\\%
\hline%
\hline%
$\chi_{b0}(3P)$&$\Upsilon(2S)$&nan&-0.09&2.55&-0.10&2.29&-0.09\\%
$\chi_{b1}(3P)$&$\Upsilon(2S)$&nan&-0.30&5.63&-0.31&5.43&-0.30\\%
$\chi_{b2}(3P)$&$\Upsilon(2S)$&nan&0.58&6.72&0.49&7.91&0.58\\%
\hline%
\hline%
\multicolumn{2}{|c|}{Subtotal}&\multicolumn{2}{|r|}{$0.18~(0.10\sigma)$}&\multicolumn{2}{|r|}{$0.07~(0.04\sigma)$}&\multicolumn{2}{|r|}{$0.18~(0.10\sigma)$}\\%
\hline%
\hline%
\multicolumn{2}{|c|}{Total}&\multicolumn{2}{|r|}{$-0.79~(-0.43\sigma)$}&\multicolumn{2}{|r|}{$-1.04~(-0.56\sigma)$}&\multicolumn{2}{|r|}{$-1.12~(-0.61\sigma)$}\\%
\hline%
\end{tabular}
\end{table}

\begin{table}[H]
    \caption{Sumrule computed for $\Upsilon(2S)$ in screened potential in E1 approximation, measured in $\mu b$. Error $\sigma \approx 1.78 \mu b$ comes from comparison to Deng et al \label{tab:srE1.yps_2S.b-scr}}
    \begin{tabular}{|l|l|c|c|c|c|c|c|}%
\hline%
&&PDG&$SR-PDG$&Deng&$SR-Deng$&E1-$\Gamma$&$SR-\Gamma$\\%
In&Out&&&&&&\\%
\hline%
$\Upsilon(2S)$&$\chi_{b0}(1P)$&1.22&-3.26&1.09&-2.92&0.89&-2.38\\%
$\Upsilon(2S)$&$\chi_{b1}(1P)$&2.21&-5.90&2.17&-5.81&1.41&-3.78\\%
$\Upsilon(2S)$&$\chi_{b2}(1P)$&2.29&9.80&2.62&11.23&1.58&6.76\\%
\hline%
\hline%
\multicolumn{2}{|c|}{Subtotal}&\multicolumn{2}{|r|}{$0.64 (1.11\sigma)$}&\multicolumn{2}{|r|}{$2.50 (4.34\sigma)$}&\multicolumn{2}{|r|}{$0.60 (1.04\sigma)$}\\%
\hline%
\hline%
$\chi_{b0}(2P)$&$\Upsilon(2S)$&nan&-5.34&14.40&-6.16&12.47&-5.34\\%
$\chi_{b1}(2P)$&$\Upsilon(2S)$&nan&-7.19&15.30&-7.31&15.05&-7.19\\%
$\chi_{b2}(2P)$&$\Upsilon(2S)$&nan&10.57&15.30&10.24&15.79&10.57\\%
\hline%
\hline%
\multicolumn{2}{|c|}{Subtotal}&\multicolumn{2}{|r|}{$-1.96 (-3.40\sigma)$}&\multicolumn{2}{|r|}{$-3.23 (-5.60\sigma)$}&\multicolumn{2}{|r|}{$-1.96 (-3.40\sigma)$}\\%
\hline%
\hline%
$\chi_{b0}(3P)$&$\Upsilon(2S)$&nan&-0.17&2.55&-0.10&4.23&-0.17\\%
$\chi_{b1}(3P)$&$\Upsilon(2S)$&nan&-0.41&5.63&-0.30&7.84&-0.41\\%
$\chi_{b2}(3P)$&$\Upsilon(2S)$&nan&0.77&6.72&0.54&9.65&0.77\\%
\hline%
\hline%
\multicolumn{2}{|c|}{Subtotal}&\multicolumn{2}{|r|}{$0.19~(0.33\sigma)$}&\multicolumn{2}{|r|}{$0.14~(0.24\sigma)$}&\multicolumn{2}{|r|}{$0.19~(0.33\sigma)$}\\%
\hline%
\hline%
\multicolumn{2}{|c|}{Total}&\multicolumn{2}{|r|}{$-1.13~(-1.96\sigma)$}&\multicolumn{2}{|r|}{$-0.59~(-1.03\sigma)$}&\multicolumn{2}{|r|}{$-1.17~(-2.03\sigma)$}\\%
\hline%
\end{tabular}
\end{table}

\begin{table}[H]
    \caption{Sumrule computed for $\Upsilon(2S)$ in screened potential in ELW approximation, measured in $\mu b$. Error $\sigma \approx 1.82 \mu b$ comes from comparison to Deng et al \label{tab:srELW.yps_2S.b-scr}}
    \begin{tabular}{|l|l|c|c|c|c|c|c|}%
\hline%
&&PDG&$SR-PDG$&Deng&$SR-Deng$&ELW-$\Gamma$&$SR-\Gamma$\\%
In&Out&&&&&&\\%
\hline%
$\Upsilon(2S)$&$\chi_{b0}(1P)$&1.22&-3.26&1.09&-2.92&0.91&-2.44\\%
$\Upsilon(2S)$&$\chi_{b1}(1P)$&2.21&-5.90&2.17&-5.81&1.43&-3.83\\%
$\Upsilon(2S)$&$\chi_{b2}(1P)$&2.29&9.80&2.62&11.23&1.59&6.83\\%
\hline%
\hline%
\multicolumn{2}{|c|}{Subtotal}&\multicolumn{2}{|r|}{$0.64 (1.47\sigma)$}&\multicolumn{2}{|r|}{$2.50 (5.73\sigma)$}&\multicolumn{2}{|r|}{$0.56 (1.27\sigma)$}\\%
\hline%
\hline%
$\chi_{b0}(2P)$&$\Upsilon(2S)$&nan&-5.70&14.40&-6.16&13.32&-5.70\\%
$\chi_{b1}(2P)$&$\Upsilon(2S)$&nan&-7.90&15.30&-7.31&16.54&-7.90\\%
$\chi_{b2}(2P)$&$\Upsilon(2S)$&nan&11.82&15.30&10.24&17.67&11.82\\%
\hline%
\hline%
\multicolumn{2}{|c|}{Subtotal}&\multicolumn{2}{|r|}{$-1.78 (-4.07\sigma)$}&\multicolumn{2}{|r|}{$-3.23 (-7.40\sigma)$}&\multicolumn{2}{|r|}{$-1.78 (-4.07\sigma)$}\\%
\hline%
\hline%
$\chi_{b0}(3P)$&$\Upsilon(2S)$&nan&-0.07&2.55&-0.10&1.69&-0.07\\%
$\chi_{b1}(3P)$&$\Upsilon(2S)$&nan&-0.24&5.63&-0.30&4.52&-0.24\\%
$\chi_{b2}(3P)$&$\Upsilon(2S)$&nan&0.50&6.72&0.54&6.26&0.50\\%
\hline%
\hline%
\multicolumn{2}{|c|}{Subtotal}&\multicolumn{2}{|r|}{$0.19~(0.45\sigma)$}&\multicolumn{2}{|r|}{$0.14~(0.32\sigma)$}&\multicolumn{2}{|r|}{$0.19~(0.45\sigma)$}\\%
\hline%
\hline%
\multicolumn{2}{|c|}{Total}&\multicolumn{2}{|r|}{$-0.94~(-2.16\sigma)$}&\multicolumn{2}{|r|}{$-0.59~(-1.36\sigma)$}&\multicolumn{2}{|r|}{$-1.03~(-2.35\sigma)$}\\%
\hline%
\end{tabular}
\end{table}

One can notice that for bottomonium there is almost no data provided by PDG, so we can only compare to Deng's results. Nevertheless, sumrule still improves when applying coefficients which take into account higher multipoles.

Finally, as a motivation for further research we provide sumrules computed for $\psi_1(1D)$ states in screened potential. They cancel not that good as those for low-lying states, but exactly these kind of states are interersting due to significant contributions from states above $D\bar{D}$ threshold.

\begin{table}[H]
    {\caption{Sumrule computed for $\psi_1(1D)$ with multipoles up to J=8 measured in $\mu b$. Error $\sigma \approx 3.7 \mu b$ comes from comparison to Deng et al \label{tab:sr.psi_1_1D.c-scr}}}
    \begin{tabular}{|l|l|c|c|c|c|c|c|}%
\hline%
&&PDG&$SR-PDG$&Deng&$SR-Deng$&$\Gamma$&$SR-\Gamma$\\%
In&Out&&&&&&\\%
\hline%
$\psi_{1}(1D)$&$\eta_{c}(1S)$&nan&-0.01&nan&-0.01&0.43&-0.01\\%
$\psi_{1}(1D)$&$\chi_{c0}(1P)$&190.40&-53.20&261.00&-72.92&288.16&-80.51\\%
$\psi_{1}(1D)$&$\chi_{c1}(1P)$&67.46&-21.21&135.00&-42.44&152.12&-47.82\\%
$\psi_{1}(1D)$&$\chi_{c2}(1P)$&nan&7.04&8.10&5.73&9.94&7.04\\%
\hline%
\hline%
\multicolumn{2}{|c|}{Subtotal}&\multicolumn{2}{|r|}{$-67.38 (-11.23\sigma)$}&\multicolumn{2}{|r|}{$-109.64 (-18.27\sigma)$}&\multicolumn{2}{|r|}{$-121.31 (-20.22\sigma)$}\\%
\hline%
\hline%
$\psi_{1}(1D)$&$\eta_{c}(2S)$&nan&-0.00&nan&-0.00&0.00&-0.00\\%
$\chi_{c0}(2P)$&$\psi_{1}(1D)$&nan&-10.15&12.00&-17.74&6.87&-10.15\\%
$\chi_{c1}(2P)$&$\psi_{1}(1D)$&nan&-24.77&9.80&-24.58&9.88&-24.77\\%
$\chi_{c2}(2P)$&$\psi_{1}(1D)$&nan&1.00&0.46&0.99&0.47&1.00\\%
\hline%
\hline%
\multicolumn{2}{|c|}{Subtotal}&\multicolumn{2}{|r|}{$-33.93 (-5.65\sigma)$}&\multicolumn{2}{|r|}{$-41.33 (-6.89\sigma)$}&\multicolumn{2}{|r|}{$-33.93 (-5.65\sigma)$}\\%
\hline%
\hline%
$\chi_{c0}(3P)$&$\psi_{1}(1D)$&nan&-0.63&0.39&-0.03&7.14&-0.63\\%
$\chi_{c1}(3P)$&$\psi_{1}(1D)$&nan&-0.95&2.00&-0.20&9.42&-0.95\\%
$\chi_{c2}(3P)$&$\psi_{1}(1D)$&nan&-0.02&0.79&-0.01&1.21&-0.02\\%
\hline%
\hline%
\multicolumn{2}{|c|}{Subtotal}&\multicolumn{2}{|r|}{$-1.60~(-0.27\sigma)$}&\multicolumn{2}{|r|}{$-0.25~(-0.04\sigma)$}&\multicolumn{2}{|r|}{$-1.60~(-0.27\sigma)$}\\%
\hline%
\hline%
\multicolumn{2}{|c|}{Total}&\multicolumn{2}{|r|}{$-102.91~(-17.15\sigma)$}&\multicolumn{2}{|r|}{$-151.22~(-25.20\sigma)$}&\multicolumn{2}{|r|}{$-156.84~(-26.14\sigma)$}\\%
\hline%
\end{tabular}
\end{table}

\begin{table}[H]
    {\caption{Sumrule computed for $\psi_1(1D)$ in E1 approximation, measured in $\mu b$. Error $\sigma \approx 5 \mu b$ comes from comparison to Deng et al \label{tab:srE1.psi_1_1D.c-scr}}}
    \begin{tabular}{|l|l|c|c|c|c|c|c|}%
\hline%
&&PDG&$SR-PDG$&Deng&$SR-Deng$&E1-$\Gamma$&$SR-\Gamma$\\%
In&Out&&&&&&\\%
\hline%
$\psi_{1}(1D)$&$\eta_{c}(1S)$&nan&-0.01&nan&-0.01&0.21&-0.01\\%
$\psi_{1}(1D)$&$\chi_{c0}(1P)$&190.40&-53.20&261.00&-72.92&253.30&-70.77\\%
$\psi_{1}(1D)$&$\chi_{c1}(1P)$&67.46&-22.76&135.00&-45.54&126.75&-42.76\\%
$\psi_{1}(1D)$&$\chi_{c2}(1P)$&nan&4.05&8.10&4.67&7.02&4.05\\%
\hline%
\hline%
\multicolumn{2}{|c|}{Subtotal}&\multicolumn{2}{|r|}{$-71.92 (-8.99\sigma)$}&\multicolumn{2}{|r|}{$-113.81 (-14.23\sigma)$}&\multicolumn{2}{|r|}{$-109.50 (-13.69\sigma)$}\\%
\hline%
\hline%
$\psi_{1}(1D)$&$\eta_{c}(2S)$&nan&-0.00&nan&-0.00&0.00&-0.00\\%
$\chi_{c0}(2P)$&$\psi_{1}(1D)$&nan&-10.36&12.00&-17.74&7.01&-10.36\\%
$\chi_{c1}(2P)$&$\psi_{1}(1D)$&nan&-26.11&9.80&-23.74&10.78&-26.11\\%
$\chi_{c2}(2P)$&$\psi_{1}(1D)$&nan&1.71&0.46&1.42&0.56&1.71\\%
\hline%
\hline%
\multicolumn{2}{|c|}{Subtotal}&\multicolumn{2}{|r|}{$-34.76 (-4.34\sigma)$}&\multicolumn{2}{|r|}{$-40.06 (-5.01\sigma)$}&\multicolumn{2}{|r|}{$-34.76 (-4.34\sigma)$}\\%
\hline%
\hline%
$\chi_{c0}(3P)$&$\psi_{1}(1D)$&nan&-0.61&0.39&-0.03&6.91&-0.61\\%
$\chi_{c1}(3P)$&$\psi_{1}(1D)$&nan&-1.02&2.00&-0.19&10.79&-1.02\\%
$\chi_{c2}(3P)$&$\psi_{1}(1D)$&nan&0.08&0.79&0.11&0.55&0.08\\%
\hline%
\hline%
\multicolumn{2}{|c|}{Subtotal}&\multicolumn{2}{|r|}{$-1.55~(-0.19\sigma)$}&\multicolumn{2}{|r|}{$-0.11~(-0.01\sigma)$}&\multicolumn{2}{|r|}{$-1.55~(-0.19\sigma)$}\\%
\hline%
\hline%
\multicolumn{2}{|c|}{Total}&\multicolumn{2}{|r|}{$-108.22~(-13.53\sigma)$}&\multicolumn{2}{|r|}{$-153.98~(-19.25\sigma)$}&\multicolumn{2}{|r|}{$-145.80~(-18.23\sigma)$}\\%
\hline%
\end{tabular}
\end{table}

\begin{table}[H]
    {\caption{Sumrule computed for $\psi_1(1D)$ in ELW approximation, measured in $\mu b$. Error $\sigma \approx 9 \mu b$ comes from comparison to Deng et al \label{tab:srELW.psi_1_1D.c-scr}}}
    \begin{tabular}{l|l|c|c|c|c|c|c}
\toprule
                &                &  PDG & $SR-PDG$ &  Deng & $SR-Deng$ &  ELW-$\Gamma$ & $SR-\Gamma$ \\
\textbf{In} & \textbf{Out} &      &          &       &           &               &             \\
\midrule
\textbf{$\psi_{1}(1D)$} & \textbf{$\eta_{c}(2S)$} &  nan &      NaN &   nan &       NaN &             0 &         NaN \\
                & \textbf{$\eta_{c}(1S)$} &  nan &      NaN &   nan &       NaN &             0 &         NaN \\
                & \textbf{$\chi_{c2}(1P)$} &  nan &     4.73 &   8.1 &      4.67 &          8.22 &        4.73 \\
                & \textbf{$\chi_{c1}(1P)$} & 67.5 &    -22.8 &   135 &     -45.5 &           153 &       -51.8 \\
                & \textbf{$\chi_{c0}(1P)$} &  190 &    -53.2 &   261 &     -72.9 &           349 &       -97.4 \\
\textbf{$\chi_{c0}(3P)$} & \textbf{$\psi_{1}(1D)$} &  nan &   -0.746 &  0.39 &   -0.0343 &          8.48 &      -0.746 \\
\textbf{$\chi_{c2}(3P)$} & \textbf{$\psi_{1}(1D)$} &  nan &     0.01 &  0.79 &      0.11 &        0.0719 &        0.01 \\
\textbf{$\chi_{c1}(3P)$} & \textbf{$\psi_{1}(1D)$} &  nan &  -0.0335 &     2 &    -0.189 &         0.355 &     -0.0335 \\
\textbf{$\chi_{c0}(2P)$} & \textbf{$\psi_{1}(1D)$} &  nan &    -10.5 &    12 &     -17.7 &          7.12 &       -10.5 \\
\textbf{$\chi_{c2}(2P)$} & \textbf{$\psi_{1}(1D)$} &  nan &     1.96 &  0.46 &      1.42 &         0.638 &        1.96 \\
\textbf{$\chi_{c1}(2P)$} & \textbf{$\psi_{1}(1D)$} &  nan &    -28.5 &   9.8 &     -23.7 &          11.8 &       -28.5 \\
\bottomrule
\end{tabular}

\end{table}
