\startcomponent intro

\product ananyev-master

\chapter{Introduction}
Quarks --- strongly interacting subatomic particles independently hypothesized by Murray Gell-Mann \cite[gellmann-quarks] and George Zweig \cite[zweig-quarks] in 1964. The idea described successfully all known hadrons at that time and also predicted additional states which were subsequently discovered.

According to the model, nowadays known as \quotation{Constituent quark model} (CQM), quarks are distinguished by three generations (first, second, third) and two types (with charges $-\frac{1}{3}e$ and $\frac{2}{3}e$). In CQM quarks can form $q \bar{q}$ pairs named mesons and $qqq$ triplets named baryons. Interaction between quarks is described by Quantum Chromodynamics(QCD), which is a non-abelian theory with additional $SU(3)$ charge (color) and  interaction carriers named gluons. Main features of QCD are color confinement (states that one can not observe isolated color charge) and asymptotic freedom (quarks and gluons do not interact at high energies).
While in 1960s QCD based phenomenological models showed up well in describing known hadrons, since that time more complicated states than $q\bar{q}$ and $qqq$ were discovered. They have been mentioned by Hell-Mann in his article and now show signs at experiments \cite[Xbabar,Xbelle,Ybabar]. As far as non-perturbative effects play a crucial role in low energy QCD processes, only effective models are possible to verify at experiment. At the same time, due to complex structure of new states, there are dozens of effective models possible.

In our work we are establishing a new channel between experimental observables and theoretical predictions by applying sumrules emerging from Light-by-Light scattering to radiative transitions of charmonium.

\subject{Charmonium system}
Charmonium is a meson built from $c$-quark and its anti-quark $\bar{c}$. At low energies one can apply non-relativistic Shrodinger equation to solve Coulomb-like problem and classify some low-lying states. By analogy to Coulomb problem, charmonium states are characterized by radial quantum number $n$, the relative orbital momentum between quark and anti-quark $L$, total spin of the system $S$, total angular momentum $J$ and its projections. Usually people label values of $L$ as S, P, D, F,~…  which correspond to 0, 1, 2, 3,~…. Then, a state with specific spin, orbital momentum, orbital quantum number and total angular momentum (excluding hyperfine splitting over total angular momentum projections) is labeled as $n^{(2S+1)} L_J$. There is another taxonomy, when people are interested in parities. Space parity is denoted by $P = (-1)^{L+1}$ and charge eigenvalue is $C = (-1)^{L+S}$. Together with the total angular momentum they form a symbol: $J^{P C}$.

To discuss the motivation for investigating charmonium, we present a diagram \infig[fig:charm-states] with its energy levels. There are $D\bar{D}$ thresholds shown there, which correspond to an energy of $c \bar{c}$ which is enough to decay into a pair of real mesons. Such a process is relativistic by its nature, so the threshold is treated as a limit above which states are not obliged to respect non-relativistic equations. It is important to mention that all states below $D\bar{D}$ have been already observed.

\placefigure[here][fig:charm-states]
    {The status of charmonium spectrum in August 2017. Red dashed lines represent theoretical predictions based on Godfrey-Isgur model with relativistic corrections to higher-excited states \cite[gbs-model]. Black solid lines are experimentally measured energy levels. Here are also shown measured transitions. We are interested in radiative transitions which are represented by green dashed lines: thick for E1 and thin for M1. Source: Olsen, Skwarnicki, Zieminska \cite[heavy-quark_pics] }
    { \externalfigure[charmoniumExotic.pdf][width=16cm] }

Regarding arguments for investigating charmonium. Charmonium is a two-particle system, the simplest possible system consisting of quarks. Constituent $c$-quarks possess huge mass ($\approx 1.27 GeV$) in comparison to $u$, $d$ and $s$ quarks ($\approx 0.1 GeV$). This allows us to apply non-relativistic phenomenological models for description of low-lying states. Moreover, the system is compact (we can estimate by order of magnitude $R \approx \alpha_a \cdot m_cS \approx 1 Fm$), so asymptotic freedom assumption is applicable as a boundary condition for the potentials.\cite[charm-slides]

Nevertheless, as one can see from the plot \infig[fig:charm-states], even low-lying states like $\chi_c0(1P)$ have dramatic discrepancies with experiment. This gives a clue that there is something beyond Coulomb model even in non-relativistic description of charmonium states.

\subject{Experimental progress}
Plethora of news states has become accessible for experimentalists owing to B-factories --- high-luminocity $e^+ e^-$-colliders developed for testing $CP$-violation in Standard Model. They produce large number of $B\bar{B}$ coherent pairs. There are several ways to produce charmonium from such pairs.

\stopcomponent
