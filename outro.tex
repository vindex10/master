\chapter{Conclusion}

Summation rules present a powerful approach to the high energy observables, connecting them to low-energy experimental data. This work defined main milestones and marked pitfalls on the way to establishing a sustainable channel between the frontiers. To conclude we would like to highlight main achievements of our work and then present a brief plan of what can be improved.

First of all we excellently reproduced mass spectrum of charmonium and bottomonium according to Deng et al.~\cite{deng-bot,deng-charm}. Standard deviation from experimental results is of order of $20MeV$ which is enough for our purposes. Moreover, we provided some ways of error controlling while looking for binding energies.

Then we developed a theoretical background for calculation of radiative transitions between bound states of some hamiltonian (it was non-relativistic Shrodinger equation in our case). We presented detailed derivation of multipole decomposition of electro-magnetic field and made use of it to simplify interaction hamiltonian between meson states and EM field. Thanks to such expansion we came up with a great insight, that for some transitions, in E1 approximation, coefficients representing relative contribution of particular helicities into the total decay width do not depend on radial wave function. Taking into account, that E1 contribution usually is the major one, the this fact is very useful for rough estimations.

Moreover, we prepared detailed appendices which can be helpful not only for current project but also as a general reference to such important tools like tensor operators and multipole expansion.

We also computed sumrules for different states of charmonium ($\psi(1S)$, $\psi(2S)$, $\psi_1(1D)$) and bottomonium ($\Upsilon(1S)$, $\Upsilon(2S)$). They showed a beautiful cancellation of decay widths of different processes taking part in the sumrule. It could be seen that big numbers representing those decay widths cancel to comparably small ones. The value of sumrule became better when we took into account higher multipole contributions.

We applied sumrule to the state at the $D\bar{D}$ threshold ($\psi_1(1D)$) which is of interest due to its non-Coulombian nature. The sumrule failed here, supposingly due to the fact that relativistic effects play crucial role near the threshold.

Finally, we developed a flexible and extensible code for computing mass spectrum of Shrodinger equation in non-relativistic quark model potential. One can easily modify the potential, and make use of provided Jupyter notebooks to analyse the outcome. Moreover, the code provides an interface for computing averages with obtained wave functions and for dealing with cut-offs at infinity (usually eigenfunctions gained numerically are applicable in some bound region). Developed code also can compute matrix elements and decay widths of radiative transitions between bound states described above.

In future we will improve absolute values of decay widths to fulfill experimental gaps, transition widths not presented in $PDG$. After that we are going to move above $D\bar{D}$ threshold and provide some estimations for radiative transitions for above-threshold states. Also to the list of future plans we attribute the error controlling for sumrules. Currently it is roughly estimated from the most imprecise experimental data points, but this can be done in more systematic way. Last but not least, to keep the results obtained in the work alive it is important to make use of online particle data, because it updates regularly and sumrules should keep track of that.
