\chapter{Decay widths}

As was mentioned in~\cref{sec:intro:sumrules}, sumrule gives exact zero when initial state is under threshold. For this reason we focus at $\psi(1S),\psi(2S)$ for charmonium and for testing artifacts of ``relativisticity'' $\Upsilon(1S),\Upsilon(2S)$ states of bottomonium which respect the latter condition.

We provide unpolarized decay widths and $r^{(0,2)}$ coefficients in two different approximations. $\Gamma$ represents decay width which includes electric and magnetic multipole terms up to $J=8$. Then we provide $E1$ contribution because it is a good cross-check of the rule we mentioned above, that $r$'s doesn't depend on radial wave function, as a consequence of the latter we do not acquire the error from spectrum fits. We also computed $E1$ transitions in the long-wave approximation(when $k_\gamma \rightarrow \infty$), because in that case calculations are easy to conduct by hand. We present these result under label $ELW$ in a separate table along with $E1$.

Each table shows our results in comparison to Deng et al.~\cite{deng-charm,deng-bot} and to those presented in PDG~\cite{pdg}. Unfortunately, not all decays widths are listed in PDG, in the places where there is no data we put $nan$.(\cref{tab:width:psi_1S.c-scr,tab:width:psi_1S.c-lin,tab:width:yps_1S.b-scr})

\begin{table}[H]
    \centering
    \caption{Decay width of charmonium in screened potential showed in $KeV$. States for sumrule with $\psi(1S)$ are listed \label{tab:width:psi_1S.c-scr}}
    \begin{footnotesize}
        \begin{tabular}{l|l|r|r|r|r|r|r|r|r}
\toprule
                &            &  PDG &  Deng &  $\Gamma$ &  $r^{(0)}$ &  $r^{(2)}$ &  E1-$\Gamma$ &  E1-$r^{(0)}$ &  E1-$r^{(2)}$ \\
\textbf{In} & \textbf{Out} &      &       &           &            &            &              &               &               \\
\midrule
\textbf{$\psi(1S)$} & \textbf{$\eta_{c}(1S)$} & 1.58 &  2.44 &      2.79 &          1 &          0 &          nan &           nan &           nan \\
\textbf{$\chi_{c0}(1P)$} & \textbf{$\psi(1S)$} &  151 &   179 &       142 &          1 &          0 &          142 &             1 &             0 \\
\textbf{$\chi_{c1}(1P)$} & \textbf{$\psi(1S)$} &  288 &   319 &       310 &      0.569 &          0 &          309 &           0.5 &             0 \\
\textbf{$\chi_{c2}(1P)$} & \textbf{$\psi(1S)$} &  374 &   292 &       510 &      0.143 &      0.519 &          506 &           0.1 &           0.6 \\
\textbf{$\eta_{c}(2S)$} & \textbf{$\psi(1S)$} &  nan &  2.29 &      2.91 &          1 &          0 &          nan &           nan &           nan \\
\textbf{$\chi_{c0}(2P)$} & \textbf{$\psi(1S)$} &  nan &   2.3 &      1.89 &          1 &          0 &         1.89 &             1 &             0 \\
\textbf{$\chi_{c1}(2P)$} & \textbf{$\psi(1S)$} &  nan &    88 &      74.4 &      0.623 &          0 &         73.3 &           0.5 &             0 \\
\textbf{$\chi_{c2}(2P)$} & \textbf{$\psi(1S)$} &  nan &    93 &       237 &       0.17 &      0.471 &          233 &           0.1 &           0.6 \\
\bottomrule
\end{tabular}

    \end{footnotesize}
\end{table}

\begin{table}[H]
    \centering
    \caption{Decay width of charmonium in screened potential showed in $KeV$. Comparison of $E1$ and long wave approximation($ELW$). States for sumrule with $\psi(1S)$ are listed \label{tab:width:ELW.psi_1S.c-scr}}
    \begin{footnotesize}
        \begin{tabular}{l|l|r|r|r|r|r|r|r|r}
\toprule
                &            &  PDG &  Deng &  E1-$\Gamma$ &  E1-$r^{(0)}$ &  E1-$r^{(2)}$ &  ELW-$\Gamma$ &  ELW-$r^{(0)}$ &  ELW-$r^{(2)}$ \\
\textbf{In} & \textbf{Out} &      &       &              &               &               &               &                &                \\
\midrule
\textbf{$\psi(1S)$} & \textbf{$\eta_{c}(1S)$} & 1.58 &  2.44 &         2.78 &             1 &             0 &           2.8 &              1 &              0 \\
\textbf{$\chi_{c0}(1P)$} & \textbf{$\psi(1S)$} &  133 &   179 &          167 &             1 &             0 &           187 &              1 &              0 \\
\textbf{$\chi_{c1}(1P)$} & \textbf{$\psi(1S)$} &  285 &   319 &          326 &           0.5 &             0 &           411 &            0.5 &              0 \\
\textbf{$\chi_{c2}(1P)$} & \textbf{$\psi(1S)$} &  371 &   292 &          361 &           0.1 &           0.6 &           481 &            0.1 &            0.6 \\
\textbf{$\eta_{c}(2S)$} & \textbf{$\psi(1S)$} &  nan &  2.29 &         2.91 &             1 &             0 &          9.19 &              1 &              0 \\
\textbf{$\chi_{c0}(2P)$} & \textbf{$\psi(1S)$} &  nan &   2.3 &           45 &             1 &             0 &          3.15 &              1 &              0 \\
\textbf{$\chi_{c1}(2P)$} & \textbf{$\psi(1S)$} &  nan &    88 &          206 &           0.5 &             0 &          60.1 &            0.5 &              0 \\
\textbf{$\chi_{c2}(2P)$} & \textbf{$\psi(1S)$} &  nan &    93 &          279 &           0.1 &           0.6 &           136 &            0.1 &            0.6 \\
\bottomrule
\end{tabular}

    \end{footnotesize}
\end{table}

\begin{table}[H]
    \centering
    \caption{Decay width of charmonium in linear potential showed in $KeV$. States for sumrule with $\psi(1S)$ are listed \label{tab:width:psi_1S.c-lin}}
    \begin{footnotesize}
        \begin{tabular}{l|l|r|r|r|r|r|r|r|r}
\toprule
                &            &  PDG &  Deng &  $\Gamma$ &  $r^{(0)}$ &  $r^{(2)}$ &  E1-$\Gamma$ &  E1-$r^{(0)}$ &  E1-$r^{(2)}$ \\
\textbf{In} & \textbf{Out} &      &       &           &            &            &              &               &               \\
\midrule
\textbf{$\psi(1S)$} & \textbf{$\eta_{c}(1S)$} & 1.58 &  2.39 &      2.53 &          1 &          0 &          nan &           nan &           nan \\
\textbf{$\chi_{c0}(1P)$} & \textbf{$\psi(1S)$} &  151 &   172 &       138 &          1 &          0 &          138 &             1 &             0 \\
\textbf{$\chi_{c1}(1P)$} & \textbf{$\psi(1S)$} &  288 &   306 &       301 &      0.566 &          0 &          300 &           0.5 &             0 \\
\textbf{$\chi_{c2}(1P)$} & \textbf{$\psi(1S)$} &  374 &   284 &       490 &      0.141 &      0.523 &          486 &           0.1 &           0.6 \\
\textbf{$\eta_{c}(2S)$} & \textbf{$\psi(1S)$} &  nan &  2.64 &      3.04 &          1 &          0 &          nan &           nan &           nan \\
\textbf{$\chi_{c0}(2P)$} & \textbf{$\psi(1S)$} &  nan &   6.1 &      2.66 &          1 &          0 &         2.66 &             1 &             0 \\
\textbf{$\chi_{c1}(2P)$} & \textbf{$\psi(1S)$} &  nan &    81 &      69.1 &      0.616 &          0 &         68.2 &           0.5 &             0 \\
\textbf{$\chi_{c2}(2P)$} & \textbf{$\psi(1S)$} &  nan &    93 &       231 &      0.166 &      0.477 &          227 &           0.1 &           0.6 \\
\bottomrule
\end{tabular}

    \end{footnotesize}
\end{table}

\begin{table}[H]
    \centering
    \caption{Decay width of charmonium in linear potential showed in $KeV$. Comparison of $E1$ and long wave approximation($ELW$). States for sumrule with $\psi(1S)$ are listed \label{tab:width:ELW.psi_1S.c-lin}}
    \begin{footnotesize}
        \begin{tabular}{l|l|r|r|r|r|r|r|r|r}
\toprule
                &            &  PDG &  Deng &  E1-$\Gamma$ &  E1-$r^{(0)}$ &  E1-$r^{(2)}$ &  ELW-$\Gamma$ &  ELW-$r^{(0)}$ &  ELW-$r^{(2)}$ \\
\textbf{In} & \textbf{Out} &      &       &              &               &               &               &                &                \\
\midrule
\textbf{$\psi(1S)$} & \textbf{$\eta_{c}(1S)$} & 1.58 &  2.39 &         2.57 &             1 &             0 &           nan &            nan &            nan \\
\textbf{$\chi_{c0}(1P)$} & \textbf{$\psi(1S)$} &  133 &   172 &          172 &             1 &             0 &           180 &              1 &              0 \\
\textbf{$\chi_{c1}(1P)$} & \textbf{$\psi(1S)$} &  285 &   306 &          352 &           0.5 &             0 &           380 &            0.5 &              0 \\
\textbf{$\chi_{c2}(1P)$} & \textbf{$\psi(1S)$} &  371 &   284 &          416 &           0.1 &           0.6 &           458 &            0.1 &            0.6 \\
\textbf{$\eta_{c}(2S)$} & \textbf{$\psi(1S)$} &  nan &  2.64 &         3.03 &             1 &             0 &           nan &            nan &            nan \\
\textbf{$\chi_{c0}(2P)$} & \textbf{$\psi(1S)$} &  nan &   6.1 &         4.63 &             1 &             0 &          1.71 &              1 &              0 \\
\textbf{$\chi_{c1}(2P)$} & \textbf{$\psi(1S)$} &  nan &    81 &          114 &           0.5 &             0 &            58 &            0.5 &              0 \\
\textbf{$\chi_{c2}(2P)$} & \textbf{$\psi(1S)$} &  nan &    93 &          213 &           0.1 &           0.6 &           146 &            0.1 &            0.6 \\
\bottomrule
\end{tabular}

    \end{footnotesize}
\end{table}

Indeed, $E1$ coefficients do not depend on radial wave function. For scalars and pseudo-scalars they are obviously equal $r^{(0)}=1$, because it is the only non-vanishing transition. In comparison to the total decay widths coeffficients in $E1$ approximation acquire only slight correction. Usually $r^{(0)}$ becomes suppresed and $r^{(2)}$ --- enhanced.

\begin{table}[H]
    \centering
    \caption{Decay width of bottomonium in screened potential showed in $KeV$. States for sumrule with $\Upsilon(1S)$ are listed \label{tab:width:yps_1S.b-scr}}
    \begin{footnotesize}
        \begin{tabular}{l|l|r|r|r|r|r|r|r|r}
\toprule
                &                &  PDG &  Deng &  $\Gamma$ &  $r^{(0)}$ &  $r^{(2)}$ &  E1-$\Gamma$ &  E1-$r^{(0)}$ &  E1-$r^{(2)}$ \\
\textbf{In} & \textbf{Out} &      &       &           &            &            &              &               &               \\
\midrule
\textbf{$\Upsilon(1S)$} & \textbf{$\eta_{b}(1S)$} &  nan &  0.01 &    0.0107 &          1 &          0 &          nan &           nan &           nan \\
\textbf{$\chi_{b0}(1P)$} & \textbf{$\Upsilon(1S)$} &  nan &  27.5 &      24.2 &          1 &          0 &         24.2 &             1 &             0 \\
\textbf{$\chi_{b1}(1P)$} & \textbf{$\Upsilon(1S)$} &  nan &  31.9 &      30.2 &      0.522 &          0 &         30.2 &           0.5 &             0 \\
\textbf{$\chi_{b2}(1P)$} & \textbf{$\Upsilon(1S)$} &  nan &  31.8 &      36.1 &      0.114 &      0.573 &         36.1 &           0.1 &           0.6 \\
\textbf{$\chi_{b0}(2P)$} & \textbf{$\Upsilon(1S)$} &  nan &  5.54 &      4.44 &          1 &          0 &         4.44 &             1 &             0 \\
\textbf{$\chi_{b1}(2P)$} & \textbf{$\Upsilon(1S)$} &  nan &  10.8 &      10.7 &       0.54 &          0 &         10.7 &           0.5 &             0 \\
\textbf{$\chi_{b2}(2P)$} & \textbf{$\Upsilon(1S)$} &  nan &  12.5 &        17 &      0.124 &      0.554 &         16.9 &           0.1 &           0.6 \\
\textbf{$\chi_{b0}(3P)$} & \textbf{$\Upsilon(1S)$} &  nan &  1.87 &      1.37 &          1 &          0 &         1.37 &             1 &             0 \\
\textbf{$\chi_{b1}(3P)$} & \textbf{$\Upsilon(1S)$} &  nan &  6.41 &      5.56 &      0.552 &          0 &         5.55 &           0.5 &             0 \\
\textbf{$\chi_{b2}(3P)$} & \textbf{$\Upsilon(1S)$} &  nan &  8.17 &      10.8 &       0.13 &      0.541 &         10.7 &           0.1 &           0.6 \\
\bottomrule
\end{tabular}

    \end{footnotesize}
\end{table}

\begin{table}[H]
    \centering
    \caption{Decay width of bottomonium in screened potential showed in $KeV$. Comparison of $E1$ and long wave approximation($ELW$). States for sumrule with $\Upsilon(1S)$ are listed \label{tab:width:ELW.yps_1S.b-scr}}
    \begin{footnotesize}
        \begin{tabular}{l|l|r|r|r|r|r|r|r|r}
\toprule
                &                &  PDG &  Deng &  E1-$\Gamma$ &  E1-$r^{(0)}$ &  E1-$r^{(2)}$ &  ELW-$\Gamma$ &  ELW-$r^{(0)}$ &  ELW-$r^{(2)}$ \\
\textbf{In} & \textbf{Out} &      &       &              &               &               &               &                &                \\
\midrule
\textbf{$\chi_{b0}(1P)$} & \textbf{$\Upsilon(1S)$} &  nan &  27.5 &         24.2 &             1 &             0 &          26.9 &              1 &              0 \\
\textbf{$\chi_{b1}(1P)$} & \textbf{$\Upsilon(1S)$} &  nan &  31.9 &         30.2 &           0.5 &             0 &          32.5 &            0.5 &              0 \\
\textbf{$\chi_{b2}(1P)$} & \textbf{$\Upsilon(1S)$} &  nan &  31.8 &         36.1 &           0.1 &           0.6 &          35.7 &            0.1 &            0.6 \\
\textbf{$\chi_{b0}(2P)$} & \textbf{$\Upsilon(1S)$} &  nan &  5.54 &         4.44 &             1 &             0 &          3.97 &              1 &              0 \\
\textbf{$\chi_{b1}(2P)$} & \textbf{$\Upsilon(1S)$} &  nan &  10.8 &         10.7 &           0.5 &             0 &            10 &            0.5 &              0 \\
\textbf{$\chi_{b2}(2P)$} & \textbf{$\Upsilon(1S)$} &  nan &  12.5 &         16.9 &           0.1 &           0.6 &          14.1 &            0.1 &            0.6 \\
\textbf{$\chi_{b0}(3P)$} & \textbf{$\Upsilon(1S)$} &  nan &  1.87 &         1.37 &             1 &             0 &          1.01 &              1 &              0 \\
\textbf{$\chi_{b1}(3P)$} & \textbf{$\Upsilon(1S)$} &  nan &  6.41 &         5.55 &           0.5 &             0 &          4.83 &            0.5 &              0 \\
\textbf{$\chi_{b2}(3P)$} & \textbf{$\Upsilon(1S)$} &  nan &  8.17 &         10.7 &           0.1 &           0.6 &           8.1 &            0.1 &            0.6 \\
\bottomrule
\end{tabular}

    \end{footnotesize}
\end{table}

In case of bottomonium relativistic effects are suppressed, and non-relativistic quark model works better. Nevertheless, dynamics for coefficients is the same: $r^{(0)}$'s are suppressed when higher multipoles took into acocount, and $r^{(2)}$s are enhanced.

To trace dependency on the energy level we provide the same set of tables for $\psi(2S)$ and $\Upsilon(2S)$ states.(\cref{tab:width:psi_2S.c-scr,tab:width:psi_2S.c-lin,tab:width:yps_2S.b-scr})

\begin{table}[H]
    \centering
    \caption{Decay width of charmonium in screened potential showed in $KeV$. States for sumrule with $\psi(2S)$ are listed \label{tab:width:psi_2S.c-scr}}
    \begin{footnotesize}
        \begin{tabular}{l|l|r|r|r|r|r|r|r|r}
\toprule
                &            &   PDG &  Deng &  $\Gamma$ &  $r^{(0)}$ &  $r^{(2)}$ &  E1-$\Gamma$ &  E1-$r^{(0)}$ &  E1-$r^{(2)}$ \\
\textbf{In} & \textbf{Out} &       &       &           &            &            &              &               &               \\
\midrule
\textbf{$\psi(2S)$} & \textbf{$\eta_{c}(1S)$} &  1.01 &   7.8 &      10.3 &          1 &          0 &         10.3 &             1 &             0 \\
                & \textbf{$\chi_{c0}(1P)$} &  29.6 &    22 &      24.5 &          1 &          0 &         24.5 &             1 &             0 \\
                & \textbf{$\chi_{c1}(1P)$} &  28.3 &    45 &      36.8 &        0.5 &          0 &         36.8 &           0.5 &             0 \\
                & \textbf{$\chi_{c2}(1P)$} &    27 &    46 &        41 &        0.1 &        0.6 &           41 &           0.1 &           0.6 \\
                & \textbf{$\eta_{c}(2S)$} & 0.207 &  0.19 &     0.148 &          1 &          0 &        0.148 &             1 &             0 \\
\textbf{$\chi_{c0}(2P)$} & \textbf{$\psi(2S)$} &   nan &    99 &       103 &          1 &          0 &          103 &             1 &             0 \\
\textbf{$\chi_{c1}(2P)$} & \textbf{$\psi(2S)$} &   nan &   155 &       174 &        0.5 &          0 &          174 &           0.5 &             0 \\
\textbf{$\chi_{c2}(2P)$} & \textbf{$\psi(2S)$} &   nan &   150 &       175 &        0.1 &        0.6 &          175 &           0.1 &           0.6 \\
\textbf{$\chi_{c0}(3P)$} & \textbf{$\psi(2S)$} &   nan &   9.1 &      64.7 &          1 &          0 &         64.7 &             1 &             0 \\
\textbf{$\chi_{c1}(3P)$} & \textbf{$\psi(2S)$} &   nan &    74 &       159 &        0.5 &          0 &          159 &           0.5 &             0 \\
\textbf{$\chi_{c2}(3P)$} & \textbf{$\psi(2S)$} &   nan &    76 &       178 &        0.1 &        0.6 &          178 &           0.1 &           0.6 \\
\bottomrule
\end{tabular}

    \end{footnotesize}
\end{table}

\begin{table}[H]
    \centering
    \caption{Decay width of charmonium in screened potential showed in $KeV$. Comparison of $E1$ and long wave approximation($ELW$). States for sumrule with $\psi(2S)$ are listed \label{tab:width:ELW.psi_2S.c-scr}}
    \begin{footnotesize}
        \begin{tabular}{l|l|r|r|r|r|r|r|r|r}
\toprule
                &            &  PDG &  Deng &  E1-$\Gamma$ &  E1-$r^{(0)}$ &  E1-$r^{(2)}$ &  ELW-$\Gamma$ &  ELW-$r^{(0)}$ &  ELW-$r^{(2)}$ \\
\textbf{In} & \textbf{Out} &      &       &              &               &               &               &                &                \\
\midrule
\textbf{$\psi(2S)$} & \textbf{$\chi_{c0}(1P)$} & 28.8 &    22 &         36.6 &             1 &             0 &          33.5 &              1 &              0 \\
                & \textbf{$\chi_{c1}(1P)$} & 28.7 &    45 &           56 &           0.5 &             0 &          54.7 &            0.5 &              0 \\
                & \textbf{$\chi_{c2}(1P)$} &   28 &    46 &         44.5 &           0.1 &           0.6 &          47.5 &            0.1 &            0.6 \\
\textbf{$\chi_{c0}(2P)$} & \textbf{$\psi(2S)$} &  nan &    99 &          108 &             1 &             0 &           128 &              1 &              0 \\
\textbf{$\chi_{c1}(2P)$} & \textbf{$\psi(2S)$} &  nan &   155 &          102 &           0.5 &             0 &           116 &            0.5 &              0 \\
\textbf{$\chi_{c2}(2P)$} & \textbf{$\psi(2S)$} &  nan &   150 &          200 &           0.1 &           0.6 &           207 &            0.1 &            0.6 \\
\textbf{$\chi_{c0}(3P)$} & \textbf{$\psi(2S)$} &  nan &   9.1 &         6.62 &             1 &             0 &        0.0396 &              1 &              0 \\
\textbf{$\chi_{c1}(3P)$} & \textbf{$\psi(2S)$} &  nan &    74 &         67.6 &           0.5 &             0 &          40.1 &            0.5 &              0 \\
\textbf{$\chi_{c2}(3P)$} & \textbf{$\psi(2S)$} &  nan &    76 &          134 &           0.1 &           0.6 &          75.1 &            0.1 &            0.6 \\
\bottomrule
\end{tabular}

    \end{footnotesize}
\end{table}

\begin{table}[H]
    \centering
    \caption{Decay width of charmonium in linear potential showed in $KeV$. States for sumrule with $\psi(2S)$ are listed \label{tab:width:psi_2S.c-lin}}
    \begin{footnotesize}
        \begin{tabular}{l|l|r|r|r|r|r|r|r|r}
\toprule
                &            &   PDG &  Deng &  $\Gamma$ &  $r^{(0)}$ &  $r^{(2)}$ &  E1-$\Gamma$ &  E1-$r^{(0)}$ &  E1-$r^{(2)}$ \\
\textbf{In} & \textbf{Out} &       &       &           &            &            &              &               &               \\
\midrule
\textbf{$\psi(2S)$} & \textbf{$\eta_{c}(1S)$} &  1.01 &  8.08 &      9.63 &          1 &          0 &         9.63 &             1 &             0 \\
                & \textbf{$\chi_{c0}(1P)$} &  29.6 &    22 &      33.5 &          1 &          0 &         28.6 &             1 &             0 \\
                & \textbf{$\chi_{c1}(1P)$} &  28.3 &    42 &      41.9 &      0.474 &          0 &         39.8 &           0.5 &             0 \\
                & \textbf{$\chi_{c2}(1P)$} &    27 &    38 &      38.9 &     0.0877 &      0.625 &         40.5 &           0.1 &           0.6 \\
                & \textbf{$\eta_{c}(2S)$} & 0.207 &  0.19 &     0.158 &          1 &          0 &        0.158 &             1 &             0 \\
\textbf{$\chi_{c0}(2P)$} & \textbf{$\psi(2S)$} &   nan &   121 &       120 &          1 &          0 &          136 &             1 &             0 \\
\textbf{$\chi_{c1}(2P)$} & \textbf{$\psi(2S)$} &   nan &   139 &       223 &      0.542 &          0 &          242 &           0.5 &             0 \\
\textbf{$\chi_{c2}(2P)$} & \textbf{$\psi(2S)$} &   nan &   135 &       288 &      0.127 &      0.548 &          263 &           0.1 &           0.6 \\
\textbf{$\chi_{c0}(3P)$} & \textbf{$\psi(2S)$} &   nan &    17 &      11.3 &          1 &          0 &         16.2 &             1 &             0 \\
\textbf{$\chi_{c1}(3P)$} & \textbf{$\psi(2S)$} &   nan &    94 &      90.7 &      0.596 &          0 &          108 &           0.5 &             0 \\
\textbf{$\chi_{c2}(3P)$} & \textbf{$\psi(2S)$} &   nan &    97 &       201 &      0.155 &      0.497 &          166 &           0.1 &           0.6 \\
\bottomrule
\end{tabular}

    \end{footnotesize}
\end{table}

\begin{table}[H]
    \centering
    \caption{Decay width of charmonium in linear potential showed in $KeV$. Comparison of $E1$ and long wave approximation($ELW$). States for sumrule with $\psi(2S)$ are listed \label{tab:width:ELW.psi_2S.c-lin}}
    \begin{footnotesize}
        \begin{tabular}{l|l|r|r|r|r|r|r|r|r}
\toprule
                &            &  PDG &  Deng &  E1-$\Gamma$ &  E1-$r^{(0)}$ &  E1-$r^{(2)}$ &  ELW-$\Gamma$ &  ELW-$r^{(0)}$ &  ELW-$r^{(2)}$ \\
\textbf{In} & \textbf{Out} &      &       &              &               &               &               &                &                \\
\midrule
\textbf{$\psi(2S)$} & \textbf{$\chi_{c0}(1P)$} & 28.8 &    22 &         37.2 &             1 &             0 &          34.1 &              1 &              0 \\
                & \textbf{$\chi_{c1}(1P)$} & 28.7 &    42 &         52.1 &           0.5 &             0 &            51 &            0.5 &              0 \\
                & \textbf{$\chi_{c2}(1P)$} &   28 &    38 &         41.7 &           0.1 &           0.6 &          44.4 &            0.1 &            0.6 \\
\textbf{$\chi_{c0}(2P)$} & \textbf{$\psi(2S)$} &  nan &   121 &         94.4 &             1 &             0 &           111 &              1 &              0 \\
\textbf{$\chi_{c1}(2P)$} & \textbf{$\psi(2S)$} &  nan &   139 &         90.9 &           0.5 &             0 &           102 &            0.5 &              0 \\
\textbf{$\chi_{c2}(2P)$} & \textbf{$\psi(2S)$} &  nan &   135 &          177 &           0.1 &           0.6 &           181 &            0.1 &            0.6 \\
\textbf{$\chi_{c0}(3P)$} & \textbf{$\psi(2S)$} &  nan &    17 &         10.4 &             1 &             0 &         0.466 &              1 &              0 \\
\textbf{$\chi_{c1}(3P)$} & \textbf{$\psi(2S)$} &  nan &    94 &         85.8 &           0.5 &             0 &          37.5 &            0.5 &              0 \\
\textbf{$\chi_{c2}(3P)$} & \textbf{$\psi(2S)$} &  nan &    97 &          190 &           0.1 &           0.6 &          86.7 &            0.1 &            0.6 \\
\bottomrule
\end{tabular}

    \end{footnotesize}
\end{table}

\begin{table}[H]
    \centering
    \caption{Decay width of bottomonium in screened potential showed in $KeV$. States for sumrule with $\Upsilon(2S)$ are listed \label{tab:width:yps_2S.b-scr}}
    \begin{footnotesize}
        \begin{tabular}{l|l|r|r|r|r|r|r|r|r}
\toprule
                &                &    PDG &  Deng &  $\Gamma$ &  $r^{(0)}$ &  $r^{(2)}$ &  E1-$\Gamma$ &  E1-$r^{(0)}$ &  E1-$r^{(2)}$ \\
\textbf{In} & \textbf{Out} &        &       &           &            &            &              &               &               \\
\midrule
\textbf{$\Upsilon(2S)$} & \textbf{$\eta_{b}(1S)$} & 0.0125 & 0.066 &    0.0737 &          1 &          0 &          nan &           nan &           nan \\
                & \textbf{$\chi_{b0}(1P)$} &   1.22 &  1.09 &      1.19 &          1 &          0 &         1.19 &             1 &             0 \\
                & \textbf{$\chi_{b1}(1P)$} &   2.21 &  2.17 &      2.28 &      0.493 &          0 &         2.28 &           0.5 &             0 \\
                & \textbf{$\chi_{b2}(1P)$} &   2.29 &  2.62 &      2.58 &     0.0965 &      0.607 &         2.58 &           0.1 &           0.6 \\
\textbf{$\chi_{b0}(2P)$} & \textbf{$\Upsilon(2S)$} &    nan &  14.4 &      13.2 &          1 &          0 &         13.2 &             1 &             0 \\
\textbf{$\chi_{b1}(2P)$} & \textbf{$\Upsilon(2S)$} &    nan &  15.3 &      15.3 &      0.512 &          0 &         15.3 &           0.5 &             0 \\
\textbf{$\chi_{b2}(2P)$} & \textbf{$\Upsilon(2S)$} &    nan &  15.3 &      16.7 &      0.108 &      0.585 &         16.7 &           0.1 &           0.6 \\
\textbf{$\chi_{b0}(3P)$} & \textbf{$\Upsilon(2S)$} &    nan &  2.55 &      2.18 &          1 &          0 &         2.18 &             1 &             0 \\
\textbf{$\chi_{b1}(3P)$} & \textbf{$\Upsilon(2S)$} &    nan &  5.63 &      5.04 &      0.525 &          0 &         5.03 &           0.5 &             0 \\
\textbf{$\chi_{b2}(3P)$} & \textbf{$\Upsilon(2S)$} &    nan &  6.72 &      7.51 &      0.115 &       0.57 &          7.5 &           0.1 &           0.6 \\
\bottomrule
\end{tabular}

    \end{footnotesize}
\end{table}

\begin{table}[H]
    \centering
    \caption{Decay width of bottomonium in screened potential showed in $KeV$. Comparison of $E1$ and long wave approximation($ELW$). States for sumrule with $\Upsilon(2S)$ are listed \label{tab:width:ELW.yps_2S.b-scr}}
    \begin{footnotesize}
        \begin{tabular}{l|l|r|r|r|r|r|r|r|r}
\toprule
                &                &  PDG &  Deng &  E1-$\Gamma$ &  E1-$r^{(0)}$ &  E1-$r^{(2)}$ &  ELW-$\Gamma$ &  ELW-$r^{(0)}$ &  ELW-$r^{(2)}$ \\
\textbf{In} & \textbf{Out} &      &       &              &               &               &               &                &                \\
\midrule
\textbf{$\Upsilon(2S)$} & \textbf{$\chi_{b0}(1P)$} & 1.22 &  1.09 &        0.902 &             1 &             0 &         0.909 &              1 &              0 \\
                & \textbf{$\chi_{b1}(1P)$} & 2.21 &  2.17 &         1.43 &           0.5 &             0 &          1.43 &            0.5 &              0 \\
                & \textbf{$\chi_{b2}(1P)$} & 2.29 &  2.62 &         1.59 &           0.1 &           0.6 &          1.59 &            0.1 &            0.6 \\
\textbf{$\chi_{b0}(2P)$} & \textbf{$\Upsilon(2S)$} &  nan &  14.4 &           13 &             1 &             0 &          13.3 &              1 &              0 \\
\textbf{$\chi_{b1}(2P)$} & \textbf{$\Upsilon(2S)$} &  nan &  15.3 &           16 &           0.5 &             0 &          16.5 &            0.5 &              0 \\
\textbf{$\chi_{b2}(2P)$} & \textbf{$\Upsilon(2S)$} &  nan &  15.3 &           17 &           0.1 &           0.6 &          17.7 &            0.1 &            0.6 \\
\textbf{$\chi_{b0}(3P)$} & \textbf{$\Upsilon(2S)$} &  nan &  2.55 &         2.53 &             1 &             0 &          1.69 &              1 &              0 \\
\textbf{$\chi_{b1}(3P)$} & \textbf{$\Upsilon(2S)$} &  nan &  5.63 &          5.7 &           0.5 &             0 &          4.52 &            0.5 &              0 \\
\textbf{$\chi_{b2}(3P)$} & \textbf{$\Upsilon(2S)$} &  nan &  6.72 &         7.51 &           0.1 &           0.6 &          6.26 &            0.1 &            0.6 \\
\bottomrule
\end{tabular}

    \end{footnotesize}
\end{table}

One can notice that decay widths of Deng et al. show a hierarchy in transitions of $\chi_{cJ}(1P) \leftrightarrow \psi(nS)$ and $\chi_{bJ}(1P) \leftrightarrow \Upsilon(nS)$. It can be seen that transitions involving $\chi_{c2}$($\chi_{b2}$) and $\chi_{c1}$($\chi_{b1}$) are twice wider than those involving $\chi_{c0}$($\chi_{b0}$). Our results reproduce the hierarchy. The crucial point is that the hierarchy is not present in PDG data for charmonium, but is present for bottomonium.

Finally, we can't but provide tables for $\psi_1(1D)$ which is matched to $\psi(3770)$, a well known state of charmonium right at the $D\bar{D}$ threshold, whose spectrum dramatically declines from non-relativistic quark model.(\cref{tab:width:psi_1_1D.c-scr,tab:width:psi_1_1D.c-lin})

\begin{table}[H]
    \centering
    \caption{Decay width of charmonium in screened potential showed in $KeV$. States for sumrule with $\psi_1(1D)$ ($\psi(3770)$) are listed \label{tab:width:psi_1_1D.c-scr}}
    \begin{footnotesize}
        \begin{tabular}{l|l|r|r|r|r|r|r|r|r}
\toprule
                &                &  PDG &  Deng &  $\Gamma$ &  $r^{(0)}$ &  $r^{(2)}$ &  E1-$\Gamma$ &  E1-$r^{(0)}$ &  E1-$r^{(2)}$ \\
\textbf{In} & \textbf{Out} &      &       &           &            &            &              &               &               \\
\midrule
\textbf{$\psi_{1}(1D)$} & \textbf{$\eta_{c}(1S)$} &  nan &   nan &     0.427 &          1 &          0 &        0.427 &             1 &             0 \\
                & \textbf{$\chi_{c0}(1P)$} &  190 &   261 &       275 &          1 &          0 &          311 &             1 &             0 \\
                & \textbf{$\chi_{c1}(1P)$} & 67.5 &   135 &       119 &      0.539 &          0 &          143 &           0.5 &             0 \\
                & \textbf{$\chi_{c2}(1P)$} &  nan &   8.1 &      5.74 &      0.253 &      0.531 &         7.76 &           0.1 &           0.6 \\
                & \textbf{$\eta_{c}(2S)$} &  nan &   nan &  0.000214 &          1 &          0 &     0.000214 &             1 &             0 \\
\textbf{$\chi_{c0}(2P)$} & \textbf{$\psi_{1}(1D)$} &  nan &    12 &      7.22 &          1 &          0 &         7.08 &             1 &             0 \\
\textbf{$\chi_{c1}(2P)$} & \textbf{$\psi_{1}(1D)$} &  nan &   9.8 &      12.4 &      0.484 &          0 &         11.4 &           0.5 &             0 \\
\textbf{$\chi_{c2}(2P)$} & \textbf{$\psi_{1}(1D)$} &  nan &  0.46 &     0.747 &       0.12 &      0.694 &        0.607 &           0.1 &           0.6 \\
\textbf{$\chi_{c0}(3P)$} & \textbf{$\psi_{1}(1D)$} &  nan &  0.39 &     0.627 &          1 &          0 &         0.56 &             1 &             0 \\
\textbf{$\chi_{c1}(3P)$} & \textbf{$\psi_{1}(1D)$} &  nan &     2 &      3.63 &      0.443 &          0 &         2.84 &           0.5 &             0 \\
\textbf{$\chi_{c2}(3P)$} & \textbf{$\psi_{1}(1D)$} &  nan &  0.79 &      1.26 &      0.173 &     0.0518 &        0.218 &           0.1 &           0.6 \\
\bottomrule
\end{tabular}

    \end{footnotesize}
\end{table}

\begin{table}[H]
    \centering
    \caption{Decay width of charmonium in screened potential showed in $KeV$. Comparison of $E1$ and long wave approximation($ELW$). States for sumrule with $\psi_1(1D)$ ($\psi(3770)$) are listed \label{tab:width:ELW.psi_1_1D.c-scr}}
    \begin{footnotesize}
        \begin{tabular}{l|l|r|r|r|r|r|r|r|r}
\toprule
                &                &  PDG &  Deng &  E1-$\Gamma$ &  E1-$r^{(0)}$ &  E1-$r^{(2)}$ &  ELW-$\Gamma$ &  ELW-$r^{(0)}$ &  ELW-$r^{(2)}$ \\
\textbf{In} & \textbf{Out} &      &       &              &               &               &               &                &                \\
\midrule
\textbf{$\psi_{1}(1D)$} & \textbf{$\eta_{c}(1S)$} &  nan &   nan &        0.427 &             1 &             0 &           nan &            nan &            nan \\
                & \textbf{$\chi_{c0}(1P)$} &  190 &   261 &          311 &             1 &             0 &           349 &              1 &              0 \\
                & \textbf{$\chi_{c1}(1P)$} & 67.5 &   135 &          143 &           0.5 &             0 &           153 &            0.5 &              0 \\
                & \textbf{$\chi_{c2}(1P)$} &  nan &   8.1 &         7.76 &           0.1 &           0.6 &          8.22 &            0.1 &            0.6 \\
                & \textbf{$\eta_{c}(2S)$} &  nan &   nan &     0.000214 &             1 &             0 &           nan &            nan &            nan \\
\textbf{$\chi_{c0}(2P)$} & \textbf{$\psi_{1}(1D)$} &  nan &    12 &         7.08 &             1 &             0 &          7.12 &              1 &              0 \\
\textbf{$\chi_{c1}(2P)$} & \textbf{$\psi_{1}(1D)$} &  nan &   9.8 &         11.4 &           0.5 &             0 &          11.8 &            0.5 &              0 \\
\textbf{$\chi_{c2}(2P)$} & \textbf{$\psi_{1}(1D)$} &  nan &  0.46 &        0.607 &           0.1 &           0.6 &         0.638 &            0.1 &            0.6 \\
\textbf{$\chi_{c0}(3P)$} & \textbf{$\psi_{1}(1D)$} &  nan &  0.39 &         0.56 &             1 &             0 &          8.48 &              1 &              0 \\
\textbf{$\chi_{c1}(3P)$} & \textbf{$\psi_{1}(1D)$} &  nan &     2 &         2.84 &           0.5 &             0 &         0.355 &            0.5 &              0 \\
\textbf{$\chi_{c2}(3P)$} & \textbf{$\psi_{1}(1D)$} &  nan &  0.79 &        0.218 &           0.1 &           0.6 &        0.0719 &            0.1 &            0.6 \\
\bottomrule
\end{tabular}

    \end{footnotesize}
\end{table}

\begin{table}[H]
    \centering
    \caption{Decay width of charmonium in linear potential showed in $KeV$. States for sumrule with $\psi_1(1D)$ ($\psi(3770)$) are listed \label{tab:width:psi_1_1D.c-lin}}
    \begin{footnotesize}
        \begin{tabular}{l|l|r|r|r|r|r|r|r|r}
\toprule
                &                &  PDG &  Deng &  $\Gamma$ &  $r^{(0)}$ &  $r^{(2)}$ &  E1-$\Gamma$ &  E1-$r^{(0)}$ &  E1-$r^{(2)}$ \\
\textbf{In} & \textbf{Out} &      &       &           &            &            &              &               &               \\
\midrule
\textbf{$\psi_{1}(1D)$} & \textbf{$\chi_{c0}(1P)$} &  190 &   272 &       255 &          1 &          0 &          255 &             1 &             0 \\
                & \textbf{$\chi_{c1}(1P)$} & 67.5 &   138 &       123 &        0.5 &          0 &          123 &           0.5 &             0 \\
                & \textbf{$\chi_{c2}(1P)$} &  nan &   7.1 &      6.63 &      0.175 &      0.651 &         6.44 &           0.1 &           0.6 \\
\textbf{$\chi_{c0}(2P)$} & \textbf{$\psi_{1}(1D)$} &  nan &    20 &      16.9 &          1 &          0 &         16.9 &             1 &             0 \\
\textbf{$\chi_{c1}(2P)$} & \textbf{$\psi_{1}(1D)$} &  nan &   7.9 &      15.5 &        0.5 &          0 &         15.5 &           0.5 &             0 \\
\textbf{$\chi_{c2}(2P)$} & \textbf{$\psi_{1}(1D)$} &  nan &  0.36 &     0.795 &      0.206 &      0.661 &        0.753 &           0.1 &           0.6 \\
\textbf{$\chi_{c0}(3P)$} & \textbf{$\psi_{1}(1D)$} &  nan &  0.27 &      28.3 &          1 &          0 &         28.3 &             1 &             0 \\
\textbf{$\chi_{c1}(3P)$} & \textbf{$\psi_{1}(1D)$} &  nan &   3.2 &      21.3 &        0.5 &          0 &         21.3 &           0.5 &             0 \\
\textbf{$\chi_{c2}(3P)$} & \textbf{$\psi_{1}(1D)$} &  nan &   1.5 &      2.78 &     0.0979 &     0.0688 &         1.02 &           0.1 &           0.6 \\
\bottomrule
\end{tabular}

    \end{footnotesize}
\end{table}

\begin{table}[H]
    \centering
    \caption{Decay width of charmonium in linear potential showed in $KeV$. Comparison of $E1$ and long wave approximation($ELW$). States for sumrule with $\psi_1(1D)$ ($\psi(3770)$) are listed \label{tab:width:ELW.psi_1_1D.c-lin}}
    \begin{footnotesize}
        \begin{tabular}{l|l|r|r|r|r|r|r|r|r}
\toprule
                &                &  PDG &  Deng &  E1-$\Gamma$ &  E1-$r^{(0)}$ &  E1-$r^{(2)}$ &  ELW-$\Gamma$ &  ELW-$r^{(0)}$ &  ELW-$r^{(2)}$ \\
\textbf{In} & \textbf{Out} &      &       &              &               &               &               &                &                \\
\midrule
\textbf{$\psi_{1}(1D)$} & \textbf{$\chi_{c0}(1P)$} &  190 &   272 &          255 &             1 &             0 &           339 &              1 &              0 \\
                & \textbf{$\chi_{c1}(1P)$} & 67.5 &   138 &          123 &           0.5 &             0 &           146 &            0.5 &              0 \\
                & \textbf{$\chi_{c2}(1P)$} &  nan &   7.1 &         6.44 &           0.1 &           0.6 &          7.39 &            0.1 &            0.6 \\
\textbf{$\chi_{c0}(2P)$} & \textbf{$\psi_{1}(1D)$} &  nan &    20 &         16.9 &             1 &             0 &          17.4 &              1 &              0 \\
\textbf{$\chi_{c1}(2P)$} & \textbf{$\psi_{1}(1D)$} &  nan &   7.9 &         15.5 &           0.5 &             0 &          17.4 &            0.5 &              0 \\
\textbf{$\chi_{c2}(2P)$} & \textbf{$\psi_{1}(1D)$} &  nan &  0.36 &        0.753 &           0.1 &           0.6 &         0.908 &            0.1 &            0.6 \\
\textbf{$\chi_{c0}(3P)$} & \textbf{$\psi_{1}(1D)$} &  nan &  0.27 &         28.3 &             1 &             0 &          8.75 &              1 &              0 \\
\textbf{$\chi_{c1}(3P)$} & \textbf{$\psi_{1}(1D)$} &  nan &   3.2 &         21.3 &           0.5 &             0 &         0.328 &            0.5 &              0 \\
\textbf{$\chi_{c2}(3P)$} & \textbf{$\psi_{1}(1D)$} &  nan &   1.5 &         1.02 &           0.1 &           0.6 &        0.0873 &            0.1 &            0.6 \\
\bottomrule
\end{tabular}

    \end{footnotesize}
\end{table}
