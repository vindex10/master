\section{Cross-section $2 \rightarrow 1$ in terms of decay width} \label{sec:app:crsc-dw}

It is possible to express corss-section $2 \rightarrow 1$ in terms of decay width $1 \rightarrow 2$. We use center of mass frame. For decay width there is a well known expression~\cite{pdg}:

\begin{align}
    \dd{\Gamma} = \frac{\vec{p}_{c.m.}}{32 \pi^2 m^2} \abs{\Mcal}^2 \dd{\Omega}
\end{align}

As cross-section $2 \rightarrow 1$ is unusual, we provide complete derivation from first principles:

\begin{align}
        &\sigma(2 \rightarrow 1) = \sum_{\lambda_X} \int \lips{\vec{p}_X} \frac{(2\pi)^4 \delta^{(4)}(p_V + p_\gamma - p_X)}{(2 \Ep{V}) (2 \Ep{\gamma}) j} \abs{\Mcal_{\lambda_X, \lambda_V \lambda_\gamma}(\vec{p}_V + \vec{p}_\gamma \rightarrow \vec{p}_X)}^2 = \\
        &= \frac{2\pi}{8 \Ep{X} \Ep{V} \Ep{\gamma} j} \delta(\Ep{V} + \Ep{\gamma} - \Ep{X}) \sum_{\lambda_X} \abs{\Mcal_{\lambda_X, \lambda_V \lambda_\gamma}}^2 = \left| \Ep{V} \Ep{\gamma} j = E_{c.m.} \abs{\vec{p}_{c.m.}} \right| = \\
        &= \left| \Ep{X} = E_{c.m.} = m_X \right| = \frac{\pi}{4 m_X^2 \abs{\vec{p}_{c.m.}}} \delta(\sqrt{s} - m_X) \sum_{\lambda_X} \abs{\Mcal_{\lambda_X, \lambda_V \lambda_\gamma}}^2 = \\
        &= \frac{2 \pi \sqrt{s}}{4 m_X^2 \abs{\vec{p}_{c.m.}}} \delta(s - m_X^2) \sum_{\lambda_X} \abs{\Mcal_{\lambda_X, \lambda_V \lambda_\gamma}}^2 = \left| \abs{\vec{p}_{c.m.}} = \frac{m_X^2 - m_V^2}{2 m_X} \right| =\\
        &= \frac{\pi}{m_X^2 - m_V^2} \delta(s - m_X^2) \sum_{\lambda_X} \abs{\Mcal_{\lambda_X, \lambda_V \lambda_\gamma}}^2
\end{align}

\begin{align}
        &\Gamma(1 \rightarrow 2) =  \frac{m_X^2 - m_V^2}{64 \pi^2 m_X^3} \int \dd{\Omega} \abs{\Mcal_{\lambda_V \lambda_\gamma, \lambda_X}(\vec{p}_X \rightarrow \vec{p}_V + \vec{p}_\gamma)}^2 = \\
        &= \frac{m_X^2 - m_V^2}{64 \pi^2 m_X^3} \int \dd{\Omega} \sum_{\lambda_X^{\prime \star} \lambda_X^{\prime}} D^{J_X \star}_{\lambda^{\prime \star}_X \lambda_X}(\Omega) D^{J_X}_{\lambda^{\prime}_X \lambda_X}(\Omega) \Mcal^\star_{\lambda_V \lambda_\gamma, \lambda^{\prime \star}_X} \Mcal_{\lambda_V \lambda_\gamma, \lambda_X} = \\
        &= \left| \int \dd{\Omega}D^{J_X \star}_{\lambda^{\prime \star}_X \lambda_X}(\Omega) D^{J_X}_{\lambda^{\prime}_X \lambda_X}(\Omega) = \frac{4 \pi}{2J_X + 1} \delta_{\lambda_X^\prime, \lambda_X^{\prime \star}}  \right| = \\
        &= \frac{m_X^2 - m_V^2}{16 \pi m_X^3} \frac{1}{2J_X + 1} \sum_{\lambda_X} \abs{\Mcal_{\lambda_V \lambda_\gamma, \lambda_X}}^2
\end{align}

Where $D$s stand for Wigner-D functions needed to align quantization axis of decaying particle ($X$) along direction defined by products of the decay ($V$ and $\gamma$). It is convenient because then matrix element shows angular momentum conservation explicitly. Notice, that here we didn't sum up over $X$ helicities (the process is still polarized), and averaging over $\lambda_X$ emerged from Wigner rotations automatically, that means decay width does not depend on $\lambda_X$.

Finally, we can re-express cross-section in terms of decay width:

\begin{align} \label{eq:app:crsc-dw}
    \sigma_{\lambda} = \frac{16 \pi^2 m_X^3}{(m_X^2 - m_V^2)^2} (2J_X + 1) \Gamma_{\lambda} \delta(s - m_X^2),\quad \lambda = \lambda_\gamma - \lambda_V
\end{align}

\paragraph{Subthreshold pole} Here we assumed final states being heavier then initial vector meson. If this is not true we get a modified expression for decay width, because we still want to keep it physical. Natural physical value corresponding to such a process is decay of vector meson into the bound state with photon emission. Actually, when comparing to previous formula it means interchanging of $m_V$ width $m_X$, $\lambda_V$ with $\lambda_X$, and $J_X$ with $J_V$:

\begin{align}
        \Gamma = \frac{m_V^2 - m_X^2}{16 \pi m_V^3} \frac{1}{2J_V + 1} \sum_{\lambda_V} \abs{\Mcal_{\lambda_X \lambda_\gamma, \lambda_V}}^2 \\
        \sigma = \frac{\pi}{m_X^2 - m_V^2} \delta(s - m_X^2) \sum_{\lambda_X} \abs{\Mcal_{\lambda_V \lambda_\gamma, \lambda_X}}^2
\end{align}

At the first sight, one can notice that matrix element now contributes to decay width in a completely different way comparing to cross-section. We can reduce this difference by making use of crossing symmetry for photon:

\begin{align}
    &\text{Crossing $\gamma: in \leftrightarrow out$:}\qquad &\sum_{\lambda_V} \abs{\Mcal_{\lambda_V, \lambda_\gamma \lambda_X}}^2 \rightarrow \sum_{\lambda_V} \abs{\Mcal_{\lambda_V (-\lambda_\gamma), \lambda_X}}^2 \\
\end{align}

We acquired flipping of photon helicity in comparison to matrix element contributing to the cross-section. It doesn't affect the correspondence because we sum up over helicities of photon.

\begin{align} \label{eq:app:crsc-dw-subthr}
    \sigma = -\frac{16 \pi^2 m_X^3}{(m_V^2 - m_X^2)^2} (2J_V + 1) \Gamma \delta(s - m_X^2)
\end{align}

One shouldn't worry about emerged $-$ sign, it will cancel with denominator of the sumrule: $\propto \frac{1}{s - m_V^2}$.

\paragraph{Decay width with non-relativistic normalization of states} \label{par:app:dw-nr}
The subject is useful for us because we obtain bound states of charmonium from quantum mechanics. Some definitions:

\begin{align}
    \text{QM:}\qquad &\braket{\vec{p}_i \psi_i}{\vec{p}_f \psi_f} = \delta_{i, f} (2\pi)^3 \delta(\vec{p}_i - \vec{p}_f) \\
    \text{Covariant:}\qquad &\braket{\vec{p}_i \psi_i}{\vec{p}_f \psi_f} = \delta_{i, f} (2\pi)^3 2 \Ep{i} \delta(\vec{p}_i - \vec{p}_f)
\end{align}

We use covariant normalization for EM field and QM normalization for charmonium states. Then decay width looks as follows:

\begin{align}
    &\Gamma = \int \frac{\dd[3]{\vec{p}_V}}{(2\pi)^3} \lips{\vec{p_\gamma}}  (2\pi)^4 \delta^{(4)}(p_X - p_V - p_\gamma) \abs{\Mcal_{\lambda_V \lambda_\gamma, \lambda_X}}^2 = \\
    &= \frac{1}{4 \pi^2} \int \frac{\dd[3]{\vec{p}_\gamma}}{2 \abs{\vec{p}_\gamma}} \delta(m_X - \sqrt{\vec{p}_\gamma^2 + m_i^2} - \abs{\vec{p}_\gamma}) \abs{\Mcal_{\lambda_V \lambda_\gamma, \lambda_X}}^2 = \\
    &= \frac{1}{8\pi^2} \int \abs{\vec{p}_\gamma} \dd{\abs{\vec{p}_\gamma}} \delta(m_X - \sqrt{\vec{p}_\gamma^2 + m_V^2} - \abs{\vec{p}_\gamma}) \int \dd{\Omega}\abs{\Mcal_{\lambda_V \lambda_\gamma, \lambda_X}}^2 = \\
    &= \frac{1}{8 \pi^2} \frac{\vec{p}_\gamma \sqrt{\vec{p}_\gamma^2 + m_V^2}}{m_X} \frac{4 \pi}{2J_X + 1} \sum_{\lambda_X} \abs{\Mcal_{\lambda_V \lambda_\gamma, \lambda_X}}^2 = \\
    &= \frac{1}{8 \pi} \frac{m_X^4 - m_V^4}{m_X^3} \frac{1}{2J_X + 1} \sum_{\lambda_X} \abs{\Mcal_{\lambda_V \lambda_\gamma, \lambda_X}}^2
\end{align}
